\label{导言区}

\usepackage[heading=true]{ctex}
\usepackage{commath}
\usepackage{graphicx}
\usepackage{float}
\usepackage{caption}
\usepackage{bookmark}
\usepackage{booktabs}
\usepackage{amssymb}
\usepackage{verbatim}
\usepackage{framed}
\usepackage{mdframed}
\usepackage{enumitem}
\usepackage{threeparttable}
\usepackage[perpage]{footmisc}
\usepackage{blindtext}
\usepackage{longtable}
\usepackage{multirow}
\usepackage{booktabs}
\usepackage{wallpaper}

\usepackage{zhnumber}
\renewcommand{\thechapter}{\zhnum{chapter}}
\renewcommand{\thesection}{\arabic{chapter}.\arabic{section}}
\renewcommand{\thesubsection}{\arabic{chapter}.\arabic{section}.\arabic{subsection}}

\usepackage{amsmath}
\allowdisplaybreaks

\usepackage{color}
\usepackage{xcolor}
\definecolor{linenumbercolor}{RGB}{128,138,135}     % 代码环境行号数字颜色
\definecolor{keywordcolor}{RGB}{3,95,205}           % 代码环境关键字颜色
\definecolor{commentcolor}{RGB}{34,139,34}          % 代码环境注释颜色
\definecolor{stringcolor}{RGB}{128,0,0}             % 代码环境字符串颜色
\definecolor{backgroundcolor}{RGB}{245,245,244}     % 代码背景颜色
\definecolor{cover_background_white}{RGB}{255,245,247}

\usepackage{hyperref}
\hypersetup{
    pdfpagelayout=TwoColumnLeft,
    colorlinks=true,
    linkcolor=blue,
    filecolor=blue,      
    urlcolor=blue,
}

\mdfsetup{
    linecolor=darkgray,
}

\usepackage{setspace}
\onehalfspacing
\addtolength{\parskip}{1em}

\usepackage{titlesec}

\titleformat{\chapter}[display]
{\filcenter\heiti\bfseries\fontsize{60pt}{72pt}\selectfont}
{\fontsize{45pt}{54pt}\selectfont 第\thechapter 章}
{0pt}
{\thispagestyle{empty} \ThisCenterWallPaper{1}{pic/chapter_title_background.png} \pagecolor{white}}
[\newpage] % creates the new page

\makeatletter % 这段代码能够修改前言, 说明和后记的章节标题样式, 原理未知, 为什么不修改正文章节标题未知.
\renewcommand{\@makeschapterhead}[1]{%
  \vspace*{20\p@}%
  {\parindent \z@
    \normalfont
    \interlinepenalty\@M
    \Huge \bfseries  #1\par\nobreak
    \vskip 20\p@
  }}
\makeatother

\ctexset{ % bug: 会产生字体错误, 但不影响结果.
    chapter={
        name={第,章},
        format+=\heiti
        
    },
    section={
        format+={\raggedright\Large\heiti}
    },
}

\usepackage{fancyhdr}
\pagestyle{fancy}
\fancyhf{}
\fancyhead[LO]{\bfseries\leftmark}
\fancyhead[RE]{\bfseries\rightmark}
\fancyfoot[LO,RE]{\bfseries\thepage}
\renewcommand{\headrulewidth}{0.5pt}
\renewcommand{\footrulewidth}{0pt}
\addtolength{\headheight}{0.5pt}
\fancypagestyle{plain}{
  \pagestyle{fancy}
}

\usepackage{afterpage}
\newcommand\emptypage{
    \null
    \thispagestyle{empty}
    \addtocounter{page}{-1}
    \newpage
}

\usepackage{listings}
\lstset{
    escapeinside=@@,                                        %  逃逸字符
    numbers=left,                                           %  在左侧显示行号
    numberstyle=\color[RGB]{128,138,135},                   %  设置行号格式
    basicstyle=\footnotesize,                               %  设置代码字号.
    basicstyle=\tt,                                         %  设置代码字体
    breaklines=true,                                        %  设置自动断行.
    extendedchars=true,                                     %  是否允许使用非ASCII字符; 仅适用于8位编码,不适用于UTF-8.
    extendedchars=false,                                    %  解决代码跨页时,章节标题,页眉等汉字不显示的问题
    xleftmargin=2em,xrightmargin=2em, aboveskip=1em,        %  设置边距
    tabsize=4,                                              %  设置tab空格数
    showspaces=false                                        %  不显示空格
    frame=none,                                             %  不显示背景边框
    backgroundcolor=\color[RGB]{245,245,244},               %  设定背景颜色
    keywordstyle=\color[RGB]{3,95,205},                     %  设定关键字颜色
    commentstyle=\tt\color[RGB]{34,139,34},                 %  设置代码注释的格式
    stringstyle=\color[RGB]{128,0,0},                       %  设置字符串格式
    showstringspaces=false,                                 %  不显示字符串中的空格
    language=c++,                                           %  设置语言
    morekeywords={alignas,continute,friend,register,true,alignof,decltype,goto,reinterpret_cast,try,asm,defult,if,return,typedef,auto,delete,inline,short,typeid,bool,do,int,signed,typename,break,double,long,sizeof,union,case,dynamic_cast,mutable,static,unsigned,catch,else,namespace,static_assert,using,char,enum,nestatic_cast,virtual,char16_t,char32_t,explict,noexcept,struct,void,export,nullptr,switch,volatile,class,extern,operator,template,wchar_t,const,false,private,this,while,constexpr,float,protected,thread_local,const_cast,for,public,throw,std},
}

\renewcommand{\lstlistingname}{代码片段}
\renewcommand{\lstlistlistingname}{代码片段}

\newcommand{\sla}{\texttt{\textbackslash}}
