\chapter{你好, 世界!} \label{你好, 世界!}
    用输出``Hello, World!''作为测试程序的做法最早可见于1978年出版的书籍\emph{The C Programming Language}, 而后成为了程序界的传统, 往往使用输出``Hello, World!''来作为一上手新语言的第一个程序, 以测试环境配置是否完善, 以及使用者是否已初步理解该语言的语法. 本章也将遵循此传统, 带领读者创建并运行第一个C语言程序——一个输出``Hello, World!''的程序.

    \section{IDE的安装}
        我们在前面的章节已经提到了编译器, 想要编写并运行C语言程序, 编译器显然是必不可少的. 然而实际开发中, 除编译器外, 我们还会使用文本编辑工具, 调试工具等等配套工具, 我们把配套的代码开发工具称为集成开发环境(IDE, Integrated Development Environment). 本节中我们将介绍三种常用IDE. 需要注意, 本节中的IDE都同时支持C和C++, 但我们把C语言代码写在C++文件中, 并用C++编译器编译也是完全没问题的.
        
        \subsection*{DevCpp}
            一款轻量级的简单IDE, 配置简单, 基本可以满足初学者的需求, 也被广泛应用于信息学奥林匹克竞赛. 推荐.

            安装地址: \href{https://sourceforge.net/projects/orwelldevcpp/}{https://sourceforge.net/projects/orwelldevcpp/}. 初次启动需选择配置, 除语言外全部默认即可.

        \subsection*{VSCode}
            一款轻量级, 高扩展的多语言IDE, 可以通过安装扩展或写脚本的方式自定义功能, 是目前世界范围内被最广泛使用的IDE. 但是不自带C语言编译器, 需自己安装并配置. 具体请自行研究. 推荐.

            安装地址: \href{https://code.visualstudio.com/}{https://code.visualstudio.com/}.

        \subsection*{Visual Studio}
            一款重量级, 面向工程项目的IDE, 功能强大且复杂, 初学者几乎都用不到, 且体积巨大. 不推荐.

            安装地址: \href{https://visualstudio.microsoft.com/zh-hans/vs/}{https://visualstudio.microsoft.com/zh-hans/vs/}.

        安装好IDE后, 创建新文件, 命名为\texttt{HelloWorld.c}或\texttt{HelloWorld.cpp}(.c是C语言源代码后缀, .cpp是C++源代码后缀), 在出现的界面里就可以输入代码了.

    \section{主函数和语句} \label{主函数和语句}
        C语言程序中必须有并且只有一个\texttt{main}函数, 被称为主函数. 主函数是程序的入口, 程序总会从主函数开始运行. 我们在刚刚创建的文件中输入
\begin{lstlisting}
int main(){

    return 0;
}
\end{lstlisting}
        这就是一个最简易的程序了.

        其中, 第一行\texttt{int main()}表示这里是主函数的开始, 然后用一对大括号包裹主函数的内容.

        第三行\texttt{return 0;}表示主函数结束并返回0, 它后面用一个分号作为结束. 这样以分号作为结束的代码被称为语句, 不同的语句可以完成特定的功能.

        读者尚不能理解``函数''``返回''等概念, 就暂且把这当作固定语法, 表示程序的开始和结束. 在完成第\ref{函数}章的学习后, 读者就能理解这些概念了.

    \section{头文件} \label{头文件}
        我们在代码的最开头添加上\texttt{\#include <stdio.h>}, 添加后形如:
\begin{lstlisting}
#include <stdio.h>

int main(){

    return 0;
}
\end{lstlisting}

        这个命令表示我们引用了\texttt{stdio.h}这个头文件. 头文件指的是提前编写了特定的常用功能的文件, 常常以\texttt{.h}作为后缀, 也被称为库. 其中, 被提前预置在编译器中的头文件被称为C语言标准库\footnote{参见菜鸟教程: \href{https://www.runoob.com/cprogramming/c-standard-library.html}{https://www.runoob.com/cprogramming/c-standard-library.html}.}. 我们通过引用它们, 就可以直接使用其中预先编写的内容. 例如\texttt{stdio.h}是STanDard Input and Output的缩写, 包含了输入输出常用的功能. 我们想要输出``Hello, World!'', 就需要先引用它.

        要想引用一个头文件, 我们需要在代码最开头使用\texttt{\#include}命令, 其后接着用尖括号\texttt{<>}或双引号\texttt{""}包裹的头文件名. 被引用的对象可以是被内置在编译器中的C语言标准库, 也可以是我们自定义的头文件. 我们\textbf{建议}总是用尖括号包裹C语言标准库. 由于我们不涉及多文件编程, 如何自定义头文件和双引号包裹头文件的作用我们略过.

    \section{空白符和注释} \label{空白符和注释}
        我们注意到, 我们的代码中被添加了若干个空行和缩进(按tab键输入), 但这并没有影响我们的程序. 这是因为空格, 回车和缩进被称为空白符, 多余的空白符会在编译的第一个阶段被去除, 因此, 我们可以在代码的任意地方(除了单词之间)添加任意多个空白符

        但空白符也有其作用, 我们观察下面两段程序:
\begin{lstlisting}
int main(){
    int a = 0, b = 1;
    int c = a + b;

    return 0;
}
\end{lstlisting}

\begin{lstlisting}
int main(){
     int a=0,b           =1;
    int c=a+ b;
return 0;
}
\end{lstlisting}

        两段程序语义完全一样, 但第一段程序整洁有序, 第二段程序杂乱恼人. 这说明恰当添加空白符可以使代码整齐, 干净, 并且能让逻辑结构更加清晰, 便于阅读和修改. 我们\textbf{建议}读者使用如下的空白符规则:
        \begin{itemize}
            \item 每一个进入一个大括号内部, 所有语句添加一个缩进.
            \item 运算符两端各添加一个空格.
            \item 逗号, 分号后添加一个空格.
            \item 不同逻辑部分之间用空行隔开, 但空行不多于两行.
        \end{itemize}

        此外, 我们还可以为代码添加注释. 注释以\texttt{//}开头, 直至行尾, 或以\texttt{/**/}包裹. 编译时, 它们会被当作空白符删掉, 所以我们可以在注释中书写任意内容. 例如下面的例子:
\begin{lstlisting}
/*
    这是一段示例程序.
*/

int main(){
    int a = 1; // 定义变量a, 将其赋值为1.
    return 0;
}
\end{lstlisting}

        我们可以在注释中添加对代码的解释和标注, 提高代码的可读性.

        此外, 我们在调试代码时, 遇到需要临时去掉一部分代码时, 我们可以将其注释掉, 这样在编译时它们就会被忽略. 我们也可以使用快捷键, 先框选需要被注释掉的代码, 使用\texttt{ctrl + /}即可将其注释掉, 再按一次即可去掉其注释.

    \section{警告与错误} \label{警告与错误}
        当我们出现语法错误, 程序无法正常完成编译时, 编译器将中止编译并抛出一个错误. 当我们语法正确, 但语义可能出现歧义, 或出现可能是疏忽导致的语义错误时, 编译器会正常完成编译并抛出一个警告. 错误是必须被解决的; 警告往往是需要被解决的, 如果程序员明确知道警告是个错误的警告时, 也可以忽视, 但对于初学者, 我们\textbf{建议}解决每一个警告.

        错误的警告品类十分繁杂, 读者遇到时可根据报错信息自行解决.

    \section{第一个C语言程序} \label{第一个C语言程序}
        接下来, 我们使用 \texttt{stdio.h}中提供的\texttt{printf()}函数来输出\texttt{Hello, World!}.

        我们在主函数下一行加上\texttt{printf("Hello, World!");}, 这个语句表示将\texttt{Hello, World!}输出到控制台中, 修改后形如:
\begin{lstlisting}
#include <stdio.h>

int main(){
    printf("Hello, World!\n");

    return 0;
}
\end{lstlisting}

        然后根据安装的IDE, 进行编译. 现在我们源代码同目录中应该有一个\texttt{HelloWorld.exe}文件, 双击运行.

        读者应该看到一个黑框一闪而过. 这是因为我们的程序运行的非常快, 它输出完\texttt{Hello, World!}后就立刻关闭了, 所以我们看不见输出内容. 我们需要让它在输出后等待. 现在, 我们将代码改成:
\begin{lstlisting}
#include <stdio.h>
#include <stdlib.h>

int main(){
    printf("Hello, World!\n");
    system("pause");

    return 0;
}
\end{lstlisting}

        我们引用了另一个头文件\texttt{stdlib.h}(STanDard LIBrary), 并使用了其中\texttt{system()}函数进行程序暂停. 再次编译并运行, 我们应当看见:
\lstset{
    numbers=none,
    keywordstyle=\color[RGB]{0,0,0},
    commentstyle=\color[RGB]{0,0,0},
    stringstyle=\color[RGB]{0,0,0}
}
\begin{lstlisting}
Hello, World!
按下任意键继续...
\end{lstlisting}
\lstset{
    numbers=left,
    keywordstyle=\color[RGB]{3,95,205},
    commentstyle=\color[RGB]{34,139,34},
    stringstyle=\color[RGB]{128,0,0}
}

        \texttt{system("pause");}在指南后续还会出现, 但我们将略去输出中的``按下任意键继续...''. 至此, 我们的第一个C语言程序完成了. 你好, 世界!