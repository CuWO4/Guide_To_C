\chapter{预处理命令} \label{预处理命令}
    广义上的编译固然指将代码变为可执行文件的过程, 但狭义上的编译却只是广义上的编译中的一个环节. 广义上的编译被分为预处理, 狭义上的编译, 汇编, 链接四个过程. 其中, 在预处理阶段被编译器处理的命令被称为预处理命令.

    预处理的输出仍然是代码, 但是会根据源代码中的预处理命令, 对代码进行增添, 替换, 删除. \footnote{读者若感兴趣, 可以部署gcc编译器, 使用-E指令提示编译器只进行预处理, 查看代码被预处理后的结果.} 本章中只列出了部分预处理命令.

    注意, 预处理阶段不会检查代码语法.

    \section{注释}
        注释是最简单的预处理命令, 它们会在预处理阶段被删去或改为空白符. 这也是为什么注释中的内容对程序没有影响. 例如下面的例子:
\begin{lstlisting}
/*
    * Hello! I'm a comment.
    */

int main(){
    // Nice to meet you. I'm also a comment.
    return 0;
}
\end{lstlisting}
        经过预处理会得到:
\begin{lstlisting}
int main(){

    return 0;
}
\end{lstlisting}
        
        已略去其它提示文件信息等信息的内容, 后文亦是如此.

    \section{\texttt{\#include}命令}
        \texttt{\#include}命令我们已经很熟了, 它会把目标文件展开到源文件中, 例如\texttt{\#include <stdio.h>}会把\texttt{stdio.h}这个文件中的内容展开到我们的源文件中, 仅仅\texttt{\#include <stdio.h>}一行, 在预处理后的文件中就被展开为了881行各类函数的声明, 以及其它定义和说明. 正因有了这些函数的声明, 我们才得以调用\texttt{stdio.h}中包含的函数. (这些函数的定义另在它处)

        此外, \texttt{\#include}命令后的文件名也可以用双引号包裹, 表示从同目录下开始搜索目标文件. 我们也可以引用自己的文件. 例如如果在同目录下创建两个文件\texttt{abc.h}和\texttt{test.c}, 分别写入如下代码:
\begin{lstlisting} [caption=abc.h]
How do you do?
\end{lstlisting}

\begin{lstlisting} [caption=test.c]
#include "abc.h"

int main(){
    return 0;
}
\end{lstlisting}

        那么\texttt{test.c}被预处理后将得到:
\begin{lstlisting}
How do you do?

int main(){
    return 0;
}
\end{lstlisting}

        双引号既能引用官方库, 又能引用自建库, 而尖括号只能引用官方库. 我们\textbf{建议}读者在引用标准库时使用尖括号, 在引用自己的库时使用双引号. 一方面由于双引号优先搜索同目录文件, 而尖括号优先搜索官方库所在路径, 这样做能加快编译速度; 另一方面这也是一种约定俗成, 能一眼便知哪些库是官方库, 哪些库是自建库.

    \section{\texttt{\#define}命令}
        \texttt{\#define}命令可以定义一个宏, 语法形如
        \begin{center}
        \begin{longtable}{l}
            \texttt{\#define 宏名~内容}
        \end{longtable}
        \end{center}

        在预处理时, 编译器会把所有与\texttt{宏名}相同的独立的文本, 替换为\texttt{内容}, 被称为宏展开, 注意宏展开只是简单的替换. 例如下面的例子:
\begin{lstlisting}
#include <stdio.h>
#include <stdlib.h>

#define N 10
#define IF if(

int main(){
    int arr[N] = { 0 };
    for(int i = 0; i < N; i++){
        printf("%d ", arr[i]);
    }

    IF 1){
        system("pause");
    }

    return 0;
}
\end{lstlisting}
        被预处理后将得到(省去):
\begin{lstlisting}
int main(){
    int arr[10] = { 0 };
    for(int i = 0; i < 10; i++){
        printf("%d ", arr[i]);
    }

    if( 1){
        system("pause");
    }

    return 0;
}
\end{lstlisting}

        可见文中所有的\texttt{N}都被替换成了\texttt{10}, 而所有\texttt{IF}都被替换成了\texttt{if(}. 前者好处多多, 例如我们在需要修改数组范围时, 只需在代码开头修改宏定义的值即可, 而无需全文一处一处查找替换, 我们\textbf{建议}这么做; 后者则后患无穷, 很容易导致语义混淆, 我们\textbf{不推荐}这么做.

        此外, 我们也\textbf{建议}普通的宏名使用如下命名规范:
        \begin{itemize}
            \item 全使用大写字母.
            \item 多个单词之间用下划线隔开.
        \end{itemize}

        例如\texttt{\#define HASH\_KEY 127598716}.

        当我们的代码中有需要大量使用的特殊常数, 或具有含义的特殊数据时, 我们\textbf{建议}使用宏. 虽然宏也可以展开为常用的代码段, 我们却\textbf{不建议}这么做.

    \section{\texttt{*}带参宏函数\texttt{*}}
        我们也可以为宏加上参数, 好像函数一样, 被称为带参宏函数, 但宏函数的机理还是替换而非像真正的函数一样运行. 它的语法形如:
        \begin{center}
        \begin{longtable}{l}
            \texttt{\#define 宏名(形参\hspace*{-0.25pt}1, 形参\hspace*{-0.25pt}2,...)~内容}
        \end{longtable}
        \end{center}

        其中, 宏名和包裹形参的括号之间不可以由空白符. 当展开宏时, 在宏名后加上括号, 传入实参, 即可把\texttt{内容}中的形参替换为对应实参. 例如下面的例子:
\begin{lstlisting}
#define max(a, b) (a > b)? a : b

int main(){
    int a = 5;
    max(a, 7);
}
\end{lstlisting}
        经过预处理会得到:
\begin{lstlisting}
int main(){
    int a = 5;
    (a > 7)? a : 7;
}
\end{lstlisting}

        但是由于运算符优先级的问题, 这种写法很容易导致问题, 例如下面的例子:
\begin{lstlisting}
#define max(a, b) (a > b)? a : b

int main(){
    int a = 5;
    max(a, 7) + 4;
}
\end{lstlisting}
        经过预处理会得到:
\begin{lstlisting}
int main(){
    int a = 5;
    (a > 7)? a : 7 + 4;
}
\end{lstlisting}

        我们本是需要\texttt{a}和7中大的那个加上4, 但实际的语义却是\texttt{a}和7+4中大的那个. 解决方法很简单, 只需把宏的内容用括号包裹即可:
\begin{lstlisting}
#define max(a, b) ((a > b)? a : b)

int main(){
    int a = 5;
    max(a, 7) + 4;
}
\end{lstlisting}

        我们\textbf{建议}当用带参宏函数来表示与参数的式子时, 总是这么做.

        宏函数和真正的函数有许多行为不同的地方, 这里不一一列举. 读者只需遵循宏函数本质是替换进行推演, 便可一一辨明.

        此外, 带参宏函数还有一些吊诡的应用, 例如下面的例子:
\begin{lstlisting}
#define CODE_BLOCK(name, expr, stat) name(expr){ stat }

int main(){
    int arr[10] = { 0 };
    CODE_BLOCK(for, int i = 0; i < 10; i++, CODE_BLOCK(if, i % 2 == 0, arr[i]++;))

    return 0;
}
\end{lstlisting}
        经过预处理会得到:
\begin{lstlisting}
int main(){
    int arr[10] = { 0 };
    for(int i = 0; i < 10; i++){ if(i % 2 == 0){ arr[i]++; } }

    return 0;
}
\end{lstlisting}

        这样做虽然能运行, 但代码语义很混乱. 当多个形如这样的带参宏互相嵌套时, 代码将变得非常混乱, 我们\textbf{不推荐}这么做. 但这也提醒我们, 宏函数的本质只是替换.