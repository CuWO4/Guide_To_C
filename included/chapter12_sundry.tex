\chapter{其他内容补充} \label{其他内容补充}
    \section{\texttt{sizeof()}}
        \texttt{sizeof()}可以看作一种特殊的单目运算符, 它会返回小括号中对象占用的内存大小(单位: B). 特别地, 它可以作用于数据类型等对象. 例如下面的例子:
\begin{lstlisting}
#include <stdio.h>
#include <stdlib.h>

struct test1{
    int a;
    double b;
};

int main(){

    int a = 0;
    int arr1[10] = { 0 };
    int *arr2 = (int*)malloc(10*sizeof(int));

    printf(
        "%d\n"
        "%d\n"
        "%d\n"
        "%d\n"
        "%d\n",
        sizeof(int),
        sizeof(struct test1),
        sizeof(a),
        sizeof(arr1),
        sizeof(arr2)
    );

    system("pause");

    free(arr2);

    return 0;
}
\end{lstlisting}

        输出为
\lstset{
    numbers=none,
    keywordstyle=\color[RGB]{0,0,0},
    commentstyle=\color[RGB]{0,0,0},
    stringstyle=\color[RGB]{0,0,0}
}
\begin{lstlisting}
4
16
4
40
8
\end{lstlisting}
\lstset{
    numbers=left,
    keywordstyle=\color[RGB]{3,95,205},
    commentstyle=\color[RGB]{34,139,34},
    stringstyle=\color[RGB]{128,0,0}
}

        前两个输出说明\texttt{sizeof()}可以返回数据类型的内存占用, 其中结构体\texttt{test1}占用的内存并非$4 + 8 = 12$B 与内存对齐有关. 后三个输出说明\texttt{sizeof()}可以返回变量的内存占用, 但需要注意的是, 我们把指针当数组用时, \texttt{sizeof()}仍然视其为指针, 并会返回指针的内存占用.

    \section{\texttt{typedef}}
        typedef的语法形如
        \begin{center}
        \begin{longtable}{l}
            \texttt{typedef 旧类型名~新类型名;}
        \end{longtable}
        \end{center}
        
        这将把\texttt{新类型名}视作\texttt{旧类型名}的别名. 例如下面的例子:
\begin{lstlisting}
typedef int INTEGER;

INTEGER main(){
    INTEGER a = 0;

    return 0;
}
\end{lstlisting}
        等价于
\begin{lstlisting}
int main(){
    int a = 0;

    return 0;
}
\end{lstlisting}

        我们也可以为数组起别名:
\begin{lstlisting}
typedef char CHAR_20[20];

int main(){
    CHAR_20 a = { 0 }, b = { 0 };

    return 0;
}
\end{lstlisting}
        等价于
\begin{lstlisting}
int main(){
    char a[20] = { 0 }, b[20] = { 0 };

    return 0;
}
\end{lstlisting}

        我们也可以为指针起别名:
\begin{lstlisting}
typedef int *PTR_INT;

int main(){
    PTR_INT p1, p2;

    return 0;
}
\end{lstlisting}
        等价于
\begin{lstlisting}
int main(){
    int *p1, *p2;

    return 0;
}
\end{lstlisting}

        我们也可以为结构体起别名:
\begin{lstlisting}
typedef struct test{
    int a;
    double b;
} TEST;

int main(){
    TEST stu;

    return 0;
}
\end{lstlisting}
        等价于
\begin{lstlisting}
struct test{
    int a;
    double b;
};

int main(){
    struct test stu;

    return 0;
}
\end{lstlisting}

        需要注意的是, 我们使用\texttt{typedef}定义新类型后, 就不能在使用时再在新类型前加上别的修饰词了. 例如下面的例子:
\begin{lstlisting}
typedef int INTEGER;

int main(){
    unsigned INTEGER a = 0;

    return 0;
}
\end{lstlisting}
        会报错. 这和使用\texttt{\#define}定义新类型是不同的.

    \section{随机数}
        标准库\texttt{stdlib.h}中包括了一个随机数生成模块. 其中, 我们使用\texttt{srand(unsigned)}设置随机数种子, 使用\texttt{rand()}生成一个随机数. 例如下面的例子:
\begin{lstlisting}
#include <stdio.h>
#include <stdlib.h>

int main(){
    srand(0); // 设置随机数种子为0

    for(int i = 0; i < 5; i++){
        printf("%d\n", rand());
    }

    system("pause");

    return 0;
}
\end{lstlisting}
        将输出
\lstset{
    numbers=none,
    keywordstyle=\color[RGB]{0,0,0},
    commentstyle=\color[RGB]{0,0,0},
    stringstyle=\color[RGB]{0,0,0}
}
\begin{lstlisting}
38
7719
21238
2437
8855
\end{lstlisting}
\lstset{
    numbers=left,
    keywordstyle=\color[RGB]{3,95,205},
    commentstyle=\color[RGB]{34,139,34},
    stringstyle=\color[RGB]{128,0,0}
}

        但读者可以多运行几次, 会发现输出始终为这5个数. 这是因为计算机生成的随机数是伪随机数. 它其实是根据种子确定的, 分布几乎没有规律的, 周期很大的一个周期函数. 由于它的周期很大, 并且在不知道种子和随机算法的情况下, 很难根据过往数据推知下一个随机, 所以我们可以近似认为它是随机的.

        我们可以利用标准库\texttt{time.h}中的\texttt{time(*time\_t)}函数, 它会返回和运行时间有关的一个值, 这样就可以使得程序每次运行, 生成的随机数序列各不相同. 例如下面的例子:
\begin{lstlisting}
#include <stdio.h>
#include <stdlib.h>
#include <time.h>

int main(){
    srand((unsigned)time(NULL));

    for(int i = 0; i < 5; i++){
        printf("%d\n", rand());
    }

    system("pause");

    return 0;
}
\end{lstlisting}

        但短时间内大量运行还是会导致随机序列相近. 另一种设置种子的方法是申请一字节内存, 以内存地址作为随机数种子. 例如下面的例子:
\begin{lstlisting}
#include <stdio.h>
#include <stdlib.h>

int main(){
    void *p = malloc(1);
    srand((unsigned)p);
    free(p);

    for(int i = 0; i < 5; i++){
        printf("%d\n", rand());
    }

    system("pause");

    return 0;
}
\end{lstlisting}

        这种方法随机性大大增强了.

    \section{\texttt{const}} 
        在声明变量时, 我们可以用\texttt{const}修饰, 表示这个变量可以等同于常量, 它是只读的, 不能通过赋值或其它手段修改它的值. 例如下面的例子:
\begin{lstlisting}
int main(){
    const double SQRT_5 = 2.23607;

    SQRT_5 = 2.0; // 错误.

    return 0;
}
\end{lstlisting}

        这有助于防止一些不改被修改的变量被修改. 当这种错误发生时, 编译时即会发生错误, 更容易DEBUG, 而不至于把错误拖到运行时, 增加DEBUG时间.

        \texttt{const}也可以修饰函数的形参, 来告知函数的调用者, 实参在函数运行期间不会被修改. 例如下面的例子:
\begin{lstlisting}
#include <stdio.h>
#include <stdlib.h>

int func(const int arr[]){

    return 0;
}

int main(){

    return 0;
}
\end{lstlisting}

        当\texttt{const}修饰指针时, 有两种情况.

        \begin{description}
            \item[\texttt{const int *p}] 表示\texttt{p}指向的数据是常量, \texttt{*p}不能修改; 但\texttt{p}本身的值, 即指向谁是可修改的. 此时称\texttt{p}为常量指针.
            \item[\texttt{int *const p}] 表示\texttt{p}本身是常量, \texttt{p}不能修改, 即指向谁不能修改; 但\texttt{*p}可以修改. 此时称\texttt{p}为指针常量.
        \end{description}

        此外, 形如\texttt{const int *const p;}表示\texttt{p}本身指向谁不可修改, 同时\texttt{p}指向的数据不可修改.

        但需要注意的是, 下面的代码是允许的:
\begin{lstlisting}
int main(){
    int a = 0;
    const int *p = &a;
    a = 10;

    return 0;
}
\end{lstlisting}

        也就是说常量指针指向的数据, 不一定是常量.