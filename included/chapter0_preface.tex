\chapter*{前言}
\addcontentsline{toc}{chapter}{前言} \label{前言}
\fancyhead[LO,RE]{\bfseries 前言}

    C语言(在不引起歧义的情况下简称C)以简洁优雅著称, 语法简练, 功能强大; 同时又是最早被广泛使用的高级语言, 可以毫不夸张地说, 计算机世界的地基是C语言一行行写出来的. 学习C语言, 不仅能一定程度上帮助解决日常学习工作中遇到的问题, 更重要的是能够一窥计算机的底层运行方式, 并为将来学习其它计算机领域打下稳固的基础.

    这是一份C语言入门指南. 旨在帮助没有任何基础的新手入门C语言.

    \leavevmode \\
    指南中你将学到:
    \begin{itemize}
        \item 有关编程语言和计算机技术的基础知识.
        \item 如何使用C语言编写程序并运行.
        \item 如何规范地组织代码.
        \item 一些利用代码解决问题的编程思维.
    \end{itemize}

    \noindent
    你将不会学到:
    \begin{itemize}
        \item 复杂算法, 数据结构等非语法内容.
        \item 多文件编程, 项目管理等工程代码内容.
    \end{itemize}

    \noindent
    指南受众:
    \begin{itemize}
        \item 有一定的数学基础.
        \item 对编程有兴趣.
        \item 不需要对编程有提前了解.
    \end{itemize}

    \vspace*{5pt}

    欢迎来到编程的世界.

\chapter*{说明}
\addcontentsline{toc}{chapter}{说明} \label{说明}
\fancyhead[LO,RE]{\bfseries 说明}

    指南中小节标题形如 \texttt{*}标题\texttt{*} 的表示该部分内容为较复杂或较不常用的内容, 对完成简单的程序无较大影响, 初次阅读可跳过, 这些章节也不会赘述过多细节. 这些小节被放在章的末尾.

    \vspace*{5pt}
    指南中形如这样地内容表示正文. 

    \vspace*{5pt}
    指南中形如\textbf{这样的内容}表示我对读者的建议.

    \begin{itemize}
        \item 指南中形如这样的内容表示准则.
    \end{itemize}

    \begin{description}
        \item[对象] 指南中形如这样的内容表示对对象的说明. 
    \end{description}

    \vspace*{-20pt}
        \[ \mbox{指南中形如这样的内容表示强调的式子.} \]

\begin{lstlisting} [caption=\ ,label=c_prefaceEx]
//指南中形如这样的内容表示代码. 它们在会被引用时编号.
\end{lstlisting}

% label naming:
%   c   :   code
%   l   :   line
%       +
%  chapter name
%       +
%     name
%       +
%  (serial num)

\lstset{
    numbers=none,
    keywordstyle=\color[RGB]{255,255,255},
    commentstyle=\color[RGB]{255,255,255},
    stringstyle=\color[RGB]{255,255,255}
}
\begin{lstlisting}
指南中形如这样的内容表示程序输入, 输出和信息. 指南中不会列出所有代码的输出, 读者可以自行实验.
\end{lstlisting}
\lstset{
    numbers=left,
    keywordstyle=\color[RGB]{3,95,205},    
    commentstyle=\color[RGB]{34,139,34},   
    stringstyle=\color[RGB]{128,0,0}
}

    \vspace*{5pt}
    指南中\href{https://www.baidu.com}{形如这样的内容}表示链接, 可以点击跳转到对应的图, 表, 网页, 注释, 章节或其它支持跳转的组件. 手机端可能无法使用此功能.

    \begin{mdframed}[linecolor=darkgray]
        代码中形如这样的内容表示涉及到 \texttt{*}章节\texttt{*} 内容的文本.
    \end{mdframed}

    指南中部分内容为了方便初学者理解, 在不损失主干逻辑的前提下做了简化, 一般会用括号标注, 或在脚注中补充详细内容. 同时我们也鼓励读者自行上网搜索. 我们也将在第\ref{错误和警告}节中提到搜索信息的技巧.

    指南中难免有疏漏之处, 敬请读者谅解. 如遇错漏, 望能告知: \href{mailto: wutong.tony@foxmail.com}{我的邮箱}.

    另外, 请不要阅读国内某些编程出版物. 它们是邪恶的, 错误的, 恼人的, 里面包含了大量原则性错误, 只会误导新手. 建立一个正确的编程规范是非常重要的.