\chapter*{前言与说明}
\addcontentsline{toc}{chapter}{前言与说明} \label{前言与说明}
\fancyhead[LO,RE]{\bfseries 前言与说明}

    C语言是一门最早被广泛应用的高级编程语言, 以简洁优雅著称, 语法简练, 功能强大. 与此同时, 它历史悠久, 在计算机发展史上有着举足轻重的作用, 并且如今也还在一些领域有着不可替代的作用, 继续发光发热. 本指南则旨在以简介明快的风格, 向入门级的读者介绍C语言, 并穿插以基础计算机技术, 编程规范等相关内容, 以期协助读者初步掌握C语言, 并具有一定的计算机思维. 

    \leavevmode \\
    指南中你将学到:
    \begin{itemize}
        \item 有关编程语言和计算机技术的基础知识.
        \item 如何使用C语言编写程序.
        \item 如何恰当地组织代码.
        \item 一些初等的编程思维.
    \end{itemize}

    \vspace*{5pt}

    指南中小节标题形如 \texttt{*}标题\texttt{*} 的表示该部分内容为较复杂或较不常用的内容, 对完成简单的程序无较大影响, 初次阅读可跳过. 这些小节不会描述过多细节. 这些小节被放在章的末尾.

    \vspace*{5pt}
    指南中形如这样地内容表示正文. 

    \vspace*{5pt}
    指南中形如\textbf{这样的内容}表示我对读者的建议.

    \begin{itemize}
        \item 指南中形如这样的内容表示准则或规范.
    \end{itemize}

    \begin{description}
        \item[对象] 指南中形如这样的内容表示对对象的说明. 
    \end{description}

    \vspace*{-20pt}
        \[ \mbox{指南中形如这样的内容表示强调或式子.} \]

\begin{lstlisting} [caption=\ ,label=c_prefaceEx]
//指南中形如这样的内容表示代码.
\end{lstlisting}

% label naming:
%   c   :   code
%   l   :   line
%       +
%  chapter name
%       +
%     name
%       +
%  (serial num)

\lstset{
    numbers=none,
    keywordstyle=\color[RGB]{255,255,255},
    commentstyle=\color[RGB]{255,255,255},
    stringstyle=\color[RGB]{255,255,255}
}
\begin{lstlisting}
指南中形如这样的内容表示程序输入, 输出和信息. 指南中不会列出所有代码的输出, 以期鼓励读者自行编写程序实验.
\end{lstlisting}
\lstset{
    numbers=left,
    keywordstyle=\color[RGB]{3,95,205},    
    commentstyle=\color[RGB]{34,139,34},   
    stringstyle=\color[RGB]{128,0,0}
}

    \vspace*{5pt}
    指南中形如这样的内容: \href{https://www.baidu.com}{https://www.baidu.com} 表示链接, 可以点击跳转到对应的图, 表, 网页, 注释, 章节或其它支持跳转的组件. 手机端可能无法使用此功能.

    \begin{mdframed}[linecolor=darkgray]
        代码中形如这样的内容表示涉及到 \texttt{*}章节\texttt{*} 的内容.
    \end{mdframed}

    指南中部分内容为了方便初学者理解, 在不损失主干逻辑的前提下做了简化, 常用脚注进行补充, 指南也鼓励读者自行搜索学习.

    指南中难免有疏漏之处, 敬请读者谅解. 如遇错漏, 望能告知. 我的邮箱: \href{mailto: wutong.tony@foxmail.com}{mailto: wutong.tony@foxmail.com}.

    此外, 尽可能不要阅读国内某些编程出版物, 或在购买阅读前仔细甄别. 它们是邪恶的, 错误的, 恼人的, 有大量错误, 常常误导新手.