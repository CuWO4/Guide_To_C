\chapter{小型编程项目练习推荐} \label{小型编程项目练习推荐}
    通过自行实现一些小型项目, 我们能更深刻理解编程, 也能体会编程带给我们的成就感. 下面是一些有趣的小型项目推荐, 难度递增. 但注意, 它们中的每一个可能都并不那么容易完成, 也不是所有知识我们都已经完全了解, 刚上手时把它们实现到一个令人满意的程度可能需要数周甚至一两个月. 当遇到阻碍时, 不要忘记搜索和询问的技巧.

    \subsection*{小型游戏}
        例如贪吃蛇, 俄罗斯方块, 2048, 扫雷. 我们可能需要使用休眠函数控制游戏速度, 随机函数产生随即结果, 二维数组储存地图等. 此外, windows平台的读者可以尝试了解\texttt{conio.h}头文件, 可能帮助我们实现流畅的输入输出. 

    \subsection*{计算器}
        例如算术计算器, 方程计算器等. 前者可以实现加减乘除, 指数对数, 三角函数等的高精度估算; 后者可以实现对输入的方程或方程组进行解析和求解, 这可能需要编译原理的帮助.

    \subsection*{基础算法库}
        例如排序算法, 幂函数算法, 随机算法等. 把它们重新实现一遍, 并在自己的代码中引用它们而非标准库, 可能更有成就感, 尽管我们自己的算法库可能效率和精度都不如标准库.

        我们习惯上把成型的基本算法库称为轮子, 把编写基本算法库称为造轮子. 一方面, C语言在引用别人的轮子时很容易出现符号冲突等问题, 另一方面, C语言编程社区规模也远不如现代语言编程社区, 因此C语言程序员往往在自己的项目中造轮子. 这固然是重复工作, 但也是C语言编程特色, 不得不品尝.

    \subsection*{简单的电脑病毒}
        简单的电脑病毒往往原理并不复杂, 我们可以先实现一个在没有任何防火墙保护的电脑上可以自我复制, 并造成破坏的病毒. 记得开启虚拟机实验. 试试用无害的病毒恶搞你的朋友.
    
    \subsection*{简单的人工智能算法}
        简单的人工智能其实并不像它的名字那样可怕. 读者可以从MNIST数据集上的数字识别入手. 当你真的创造出一个会学习的``婴儿'', 或许编程和你的电脑会更可爱.