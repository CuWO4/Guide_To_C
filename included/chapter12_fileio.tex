\chapter{文件输入输出} \label{文件输入输出}
    \section{打开文件}
        \texttt{FILE}是\texttt{stdio.h}中的一个结构体, 其指针变量专门用于进行文件操作, 称为文件指针. 我们称声明一个\texttt{FILE*}变量并赋值的过程为打开文件.

        打开文件的函数为\texttt{fopen()}, 它需要两个字符串参数, 第一个参数指定目标文件的相对路径, 第二个参数指定打开方式. 文件操作完毕后需要使用\texttt{fclose()}函数关闭文件. 二者总是一一对应.

        例如下面的例子:
\begin{lstlisting}
#include <stdio.h>

int main(){
    FILE *fp = fopen("test.txt", "w"); // 以写入方式打开同目录下的test.txt文件, 储存在文件指针fp中.
    fclose(fp); // 关闭fp对应的文件.

    return 0;
}
\end{lstlisting}

        其中, 打开方式有\texttt{w}, \texttt{r}, \texttt{a}三种.
        \begin{description}
            \item[\texttt{w}] 以写入方式打开, 打开后只可以写入, 不可以读取文件内容. 若文件存在, 那么将清空文件内容; 若文件不存在, 那么将新建空白文件.
            \item[\texttt{r}] 以只读方式打开, 打开后只可以读取, 不可以写入文件内容. 若文件存在, 那么将打开文件; 若文件不存在, 将返回空指针\texttt{NULL}.
            \item[\texttt{a}] 以追加方式打开, 效果同\texttt{w}, 只是不会清空文件内容, 写入的内容将追加在文件末尾. 
        \end{description}

        我们也可以在打开方式后添加\texttt{+}后缀, 使得其它效果不变, 但可以随意读写文件.

        我们也可以添加后缀\texttt{b}或\texttt{t}, 指定文件格式为二进制格式还是字符串格式. 当后缀缺乏时, 默认为\texttt{t}.

        例如\texttt{"r+b"}就表示打开文件, 若文件不存在则返回空指针\texttt{NULL}. 可以随意读写文件. 文件读写格式为二进制.

        C语言默认有两个文件指针\texttt{stdin}和\texttt{stdout}, 分别为控制台输入和输出, 尽管控制台不是文件, 但其实也可以按照文件处理.

        此外, 与字符串的最后一个元素是\texttt{\sla 0}类似, 可以认为文件的最后一个字符是\texttt{EOF}\footnote{End Of File, 文件末尾的缩写.}. EOF实际上是\texttt{stdio.h}中的一个宏.

    \section{字符和字符串读写}
        \subsection*{\texttt{fgetc()}}
            原型为\texttt{int fgetc(FILE *fp)}, 与\texttt{getc()}类似, 区别在于从文件指针\texttt{fp}指向的文件而非控制台中读取.

        \subsection*{\texttt{fputc()}}
            原型为\texttt{int fputc(int ch, FILE *fp)}, 与\texttt{putc()}类似, 区别在于向文件指针\texttt{fp}指向的文件而非控制台中写入.

        \subsection*{\texttt{fgets()}}
            原型为\texttt{char *fgets(char *str, int n, FILE *fp)}. 其中\texttt{str}为储存内容的数组, \texttt{n}为读取长度, \texttt{fp}为指向目标文件的文件指针. 注意, 读取长度包括\texttt{\sla 0}, 故实际上最多只能从\texttt{fp}指向的文件中读取\texttt{n - 1}个字符.

            \texttt{fgets()}遇到三种情况会停止读取:
            \begin{enumerate}
                \item 字符串长度到达读取长度.
                \item 读取到换行.
                \item 读取到\texttt{EOF}.
            \end{enumerate}

            其它功能与\texttt{gets()}类似.
        
        \subsection*{\texttt{fputs()}}
            原型为\texttt{int fputs(char *str, FILE *fp)}, 与\texttt{puts()}类似, 区别在于向文件指针\texttt{fp}指向的文件而非控制台中写入.

        例如, 在同目录下分别创建\texttt{abc.txt}和\texttt{test.c}两个文件, 分别写入:
\lstset{
    numbers=none,
    keywordstyle=\color[RGB]{0,0,0},
    commentstyle=\color[RGB]{0,0,0},
    stringstyle=\color[RGB]{0,0,0}
}
\begin{lstlisting}[caption=abc.txt]
THIS! IS! FIRST! LINE!!!
be water, my friend.
        --Bruce Lee
\end{lstlisting}
\lstset{
    numbers=left,
    keywordstyle=\color[RGB]{3,95,205},
    commentstyle=\color[RGB]{34,139,34},
    stringstyle=\color[RGB]{128,0,0}
}

\begin{lstlisting} [caption=test.c, label=c_fileio]
#include <stdio.h>

int main(){
    FILE *fp = fopen("abc.txt", "r");
    if(fp == NULL){
        printf("ERROR: file not found.");
        return 0;
    }

    char *str;
    fgets(str, 100, fp);
    printf("%s", str);

    printf("\n\n");

    char c;
    while((c = fgetc(fp)) != EOF){
        printf("%c", c);
    }

    return 0;
}
\end{lstlisting}

        那么输出为
\lstset{
    numbers=none,
    keywordstyle=\color[RGB]{0,0,0},
    commentstyle=\color[RGB]{0,0,0},
    stringstyle=\color[RGB]{0,0,0}
}
\begin{lstlisting}
THIS! IS! FIRST! LINE!!!


be water, my friend.
        --Bruce Lee
\end{lstlisting}
\lstset{
    numbers=left,
    keywordstyle=\color[RGB]{3,95,205},
    commentstyle=\color[RGB]{34,139,34},
    stringstyle=\color[RGB]{128,0,0}
}

    \section{格式化读写}
        \begin{sloppypar}
        \texttt{stdio.h}中还提供了两个文件读写函数\texttt{fprintf()}和\texttt{fscanf()}, 它们分别与\texttt{printf()}和\texttt{scanf()}类似, 只是第一个参数需要提供文件指针, 而输入输出也将指向\texttt{fp}指向的文件. 
        \end{sloppypar}

        但是注意, \texttt{fscanf()}不会在末尾时为待输入文本传递\texttt{EOF}, 此时我们需要\texttt{feof()}来判断, 它的参数为文件指针, 返回是否已经到达文件末尾, 若到达, 返回非零值, 否则返回0. 
        
        例如代码\ref{c_fileio}等价于:
\begin{lstlisting}
#include <stdio.h>

int main(){
    FILE *fp = fopen("abc.txt", "r");
    if(fp == NULL){
        printf("ERROR: file not found.");
        return 0;
    }

    char *str;
    fscanf(fp, "%s", str);
    printf("%s", str);

    printf("\n\n");

    int c;
    while(1){
        fscanf(fp, "%c", &c);
        if(feof(fp) != 0){
            break;
        }
        printf("%c", c);
    }

    return 0;
}
\end{lstlisting}