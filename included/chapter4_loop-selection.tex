\chapter{分支与循环} \label{分支与循环}
    我们前文的几乎所有代码都是从\texttt{main()}开始, 一句一句顺序执行的, 这样的结构被称为顺序结构. 但是只有顺序结构的程序是很无聊的, 首先它无法进行逻辑判断, 它只能顺着预定的轨道执行下去, 直到结束; 其次我们需要它做什么, 我们就必须在对应位置写一句语句, 例如我们需要它输出100个a, 我们就必须写一百次 \texttt{printf("a");} . 本节中将介绍另外两种程序结构, 使得我们的程序真正开始有趣起来.

    \section{分支结构} % 模与奇偶, 位运算与奇偶, 逗号运算
        分支结构指的是程序运行到此处时, 将从多个分支中选择一个继续执行, 而选择的依据是我们定义的逻辑条件. 这在C语言中由两种语句实现.

        \subsection*{\texttt{if else}语句}
            分支结构主要由\texttt{if else}语句实现. 它最基础的结构形如
            \begin{align*}
                &\mbox{\texttt{if(条件)\{}}\\
                &\qquad\mbox{\texttt{语句块}}\\    
                &\mbox{\texttt{\}}}
            \end{align*}

            其中 \texttt{条件} 是一个表达式, 当 \texttt{条件} 为真时, 运行大括号内 \texttt{语句块} 的内容, 然后继续顺序运行. 否则跳过整段代码. 例如下面的例子:
\begin{lstlisting}
#include <stdio.h>
#include <stdlib.h>

int main(){
    int temp;
    scanf("%d", &temp);

    if(temp > 0){
        temp += 5;
        printf("beautiful.\n");
    }

    printf("%d\n", temp);
    
    system("pause");

    return 0;
}
\end{lstlisting}

            例如我们输入3, 那么 \texttt{temp > 0} 为真, 执行\texttt{if}后打括号内的语句, \texttt{temp} 先自增\texttt{5}变为\texttt{8}, 然后输出 \texttt{beautiful.} 大括号执行结束, 继续执行打印 \texttt{temp} , 输出8. 故输出为
\lstset{
    numbers=none,
    keywordstyle=\color[RGB]{0,0,0},
    commentstyle=\color[RGB]{0,0,0},
    stringstyle=\color[RGB]{0,0,0}
}
\begin{lstlisting}
beautiful.
8
\end{lstlisting}
\lstset{
    numbers=left,
    keywordstyle=\color[RGB]{3,95,205},
    commentstyle=\color[RGB]{34,139,34},
    stringstyle=\color[RGB]{128,0,0}
}   
            再例如我们输入-1, 那么 \texttt{temp > 0} 为假, 跳过\texttt{if}后大括号内的所有语句, 执行打印 \texttt{temp} , 输出-1. 故输出为
\lstset{
    numbers=none,
    keywordstyle=\color[RGB]{0,0,0},
    commentstyle=\color[RGB]{0,0,0},
    stringstyle=\color[RGB]{0,0,0}
}
\begin{lstlisting}
-1
\end{lstlisting}
\lstset{
    numbers=left,
    keywordstyle=\color[RGB]{3,95,205},
    commentstyle=\color[RGB]{34,139,34},
    stringstyle=\color[RGB]{128,0,0}
}

            特别地, 当 语句块 中只有一个语句时, 大括号可以省略. 例如下面的例子:
\begin{lstlisting}
#include <stdio.h>
#include <stdlib.h>

int main(){
    int temp = 0;
    scanf("%d", &temp);

    if(temp > 0) printf("beautiful.\n");

    printf("%d\n", temp);
    
    system("pause");

    return 0;
}
\end{lstlisting}

            但我们\textbf{不推荐}这么做.

            我们也可以在\texttt{if}的大括号后加入若干个\texttt{else if}来不断检验条件, 直到条件为真, 执行对应条件的语句块, 执行完毕后结束整个\texttt{if}以及连带的\texttt{else if}的检验, 继续执行后面的代码. 语法形如
            \begin{align*}
                &\mbox{\texttt{if(条件\hspace*{-0.25pt}1)\{}} \\
                &\qquad\mbox{\texttt{语句块\hspace*{-0.25pt}1}} \\    
                &\mbox{\texttt{\}else if(条件\hspace*{-0.25pt}2)\{}} \\
                &\qquad\mbox{\texttt{语句块\hspace*{-0.25pt}2}} \\
                &\mbox{\texttt{\}}} \\
                &\dots \\
                &\mbox{\texttt{else if(条件\hspace*{-0.25pt}n)\{}} \\
                &\qquad\mbox{\texttt{语句块\hspace*{-0.25pt}n}} \\
                &\mbox{\texttt{\}}} \\
            \end{align*}

            例如下面的例子:
\begin{lstlisting}
#include <stdio.h>
#include <stdlib.h>

int main(){
    int temp = 0;
    scanf("%d", &temp);

    if(temp > 5){
        printf("A\n");
    }else if(temp <= 5 && temp > 0){
        printf("B\n");
    }else if(temp < -5){
        printf("C\n");
    }

    printf("kk\n");
    
    system("pause");

    return 0;
}
\end{lstlisting}

            例如我们输入3, 第一个\texttt{if}条件为假, 检测第二个\texttt{if}, 第一个\texttt{if}条件为真, 打印\texttt{B}, 结束\texttt{if}及其连带的语句, 执行打印\texttt{kk}. 故输出为
\lstset{
    numbers=none,
    keywordstyle=\color[RGB]{0,0,0},
    commentstyle=\color[RGB]{0,0,0},
    stringstyle=\color[RGB]{0,0,0}
}
\begin{lstlisting}
B
kk
\end{lstlisting}
\lstset{
    numbers=left,
    keywordstyle=\color[RGB]{3,95,205},
    commentstyle=\color[RGB]{34,139,34},
    stringstyle=\color[RGB]{128,0,0}
}

            再例如我们输入-1, 三个\texttt{if}的条件都为假, 继续执行打印\texttt{kk}. 故输出为
\lstset{
    numbers=none,
    keywordstyle=\color[RGB]{0,0,0},
    commentstyle=\color[RGB]{0,0,0},
    stringstyle=\color[RGB]{0,0,0}
}
\begin{lstlisting}
kk
\end{lstlisting}
\lstset{
    numbers=left,
    keywordstyle=\color[RGB]{3,95,205},
    commentstyle=\color[RGB]{34,139,34},
    stringstyle=\color[RGB]{128,0,0}
}

            另外, 上面这段代码第10行中的条件可以写作\texttt{temp > 0}, 因为如果能执行这次判断, 上一次判断结果一定为假, 而\texttt{temp > 5}为假就一定已经满足$\texttt{temp} \leq 5 $. 故可以省略.

            此外, 我们也可以在一段\texttt{if}结构的最后加上一个\texttt{else}, 表示当所有条件为假时, 则执行\texttt{else}后大括号的语句. 语法形如
            \begin{align*}
                &\mbox{\texttt{if(表达式\hspace*{-0.25pt}1)\{}} \\
                &\qquad\mbox{\texttt{语句块\hspace*{-0.25pt}1}} \\    
                &\mbox{\texttt{\}else if(表达式\hspace*{-0.25pt}2)\{}} \\
                &\qquad\mbox{\texttt{语句块\hspace*{-0.25pt}2}} \\
                &\mbox{\texttt{\}}} \\
                &\dots \\
                &\mbox{\texttt{else if(表达式\hspace*{-0.25pt}n)\{}} \\
                &\qquad\mbox{\texttt{语句块\hspace*{-0.25pt}n}} \\
                &\mbox{\texttt{\}else\{}} \\
                &\qquad\mbox{\texttt{语句块\hspace*{-0.25pt}default}} \\
                &\mbox{\texttt{\}}}
            \end{align*}

            例如下面的例子:
\begin{lstlisting}
#include <stdio.h>
#include <stdlib.h>

int main(){
    int temp = 0;
    scanf("%d", &temp);

    if(temp > 5){
        printf("A\n");
    }else if(temp > 0){
        printf("B\n");
    }else if(temp < -5){
        printf("C\n");
    }else{
        printf("D\n");
    }

    printf("kk\n");
    
    system("pause");

    return 0;
}
\end{lstlisting}

            我们再输入\texttt{-1}, 三个\texttt{if}的条件均为假, 执行\texttt{else}中的语句, 打印D, 结束\texttt{if}语句, 打印kk. 故输出为
\lstset{
    numbers=none,
    keywordstyle=\color[RGB]{0,0,0},
    commentstyle=\color[RGB]{0,0,0},
    stringstyle=\color[RGB]{0,0,0}
}
\begin{lstlisting}
D
kk
\end{lstlisting}
\lstset{
    numbers=left,
    keywordstyle=\color[RGB]{3,95,205},
    commentstyle=\color[RGB]{34,139,34},
    stringstyle=\color[RGB]{128,0,0}
}

            我们\textbf{建议}读者使用下面的规范组织\texttt{if}语句:
            \begin{itemize}
                \item 当\texttt{if}后只有一个语句时, 不省略大括号.
                \item 当\texttt{else if}和\texttt{else}被夹在前后两个语句块的大括号中时, 不换行.
            \end{itemize}

            \begin{mdframed}[linecolor=darkgray]
            有时我们需要先输入一个数, 然后立即对这个数进行判断, 例如下面的例子
\begin{lstlisting}
#include <stdio.h>
#include <stdlib.h>

int main(){
    int n = 0;

    // do something

    scanf("%d", &n);
    if(n > 0){
        printf("greater than 0 !\n");
    }else if(n == 0){
        printf("equalt to 0 !\n");
    }else{
        printf("less than 0 !\n");
    }

    system("pause");

    return 0;
}
\end{lstlisting}

            这样的代码结构较为松散, 不能一眼看出代码意图. 这里介绍一种基于逗号运算符(见第\ref{逗号运算}节)的技巧. 我们可以把上面的代码写作:
\begin{lstlisting}
#include <stdio.h>
#include <stdlib.h>

int main(){
    int n = 0;

    // do something

    if(scanf("%d", &n), n > 0){
        printf("greater than 0 !\n");
    }else if(n == 0){
        printf("equalt to 0 !\n");
    }else{
        printf("less than 0 !\n");
    }

    system("pause");

    return 0;
}
\end{lstlisting}

            由于\texttt{scanf()}也可以看作一个表达式(见第\ref{函数}章), 故语法上成立. 逗号运算符会先执行\texttt{scanf()}, 再判断n > 0的真假, 然后返回n > 0的结果. 这样我们代码结构就更紧凑了.
            \end{mdframed}


        \subsection*{\texttt{switch case}语句}
            \texttt{switch}语句也可以实现分支结构, 但功能较弱. \texttt{switch}语句语法形如
                \begin{center}
                \begin{longtable}{l}
                    \texttt{switch(变量)\{} \\
                    \qquad \texttt{case 值\hspace*{-0.25pt}1:} \\
                    \qquad \qquad \texttt{语句块\hspace*{-0.25pt}1} \\
                    \qquad \qquad \texttt{break;} \\
                    \qquad \texttt{case 值\hspace*{-0.25pt}2:} \\
                    \qquad \qquad \texttt{语句块\hspace*{-0.25pt}2} \\
                    \qquad \qquad \texttt{break;} \\
                    \qquad \dots \\
                    \qquad \texttt{case 值\hspace*{-0.25pt}\texttt{n}:} \\
                    \qquad \qquad \texttt{语句块\hspace*{-0.25pt}\texttt{n}} \\
                    \qquad \qquad \texttt{break;} \\
                    \qquad \texttt{default:} \\
                    \qquad \qquad \texttt{语句块\hspace*{-0.25pt}\texttt{default}} \\
                    \texttt{\}}
                \end{longtable}
                \end{center}

                \texttt{case}可以有若干个, \texttt{default}可以省去. 注意不要漏写\texttt{break}, 否则在一个分支判定成功执行后, 将顺序执行后续分支中的代码. 
                
                上面的语法等价于
                \begin{center}
                \begin{longtable}{l}
                    \texttt{if(变量 == 值\hspace*{-0.25pt}\texttt{1})\{} \\
                    \qquad \texttt{语句块\hspace*{-0.25pt}\texttt{1}} \\
                    \texttt{\}else if(变量 == 值\hspace*{-0.25pt}\texttt{2})\{} \\
                    \qquad \texttt{语句块\hspace*{-0.25pt}\texttt{2}} \\
                    \texttt{\}} \\
                    \dots \\
                    \texttt{else if(变量 == 值\hspace*{-0.25pt}\texttt{n})\{} \\
                    \qquad \texttt{语句块\hspace*{-0.25pt}\texttt{n}} \\
                    \texttt{\}else\{} \\
                    \qquad \texttt{语句块\hspace*{-0.25pt}\texttt{default}} \\
                    \texttt{\}}
                \end{longtable}
                \end{center}

                例如下面的例子
\begin{lstlisting}
#include <stdio.h>
#include <stdlib.h>

int main(){
    int a = 0;
    scanf("%d", &a);

    switch(a){
        case 0:
            printf("I\n");
            break;
        case 1:
            printf("Love\n");
            break;
        case 2:
            printf("U\n");
            break;
        default:
            printf("ERROR: Your love is out of range!\n");
    }

    return 0;
}
\end{lstlisting}

                读者可以根据语义自行测试.

    \section{实例: 判断奇偶性}
        我们来举一个例子, 来看一看分支结构的丰富内涵, 并学习一下分析问题和优化代码的大致过程.

        输入一个整数\texttt{a}, 判断它的奇偶性. 若为奇, 输出\texttt{odd!}; 若为偶, 输出\texttt{even!}.

        我们注意到, 奇数和偶数的一个区别是能不能被2整除. 奇数被2除时, 总是余1; 偶数被2除时, 总是余0. 故我们可以使用模运算来得出\texttt{a}除2的余数, 并检验它的值来判断\texttt{a}的奇偶性. 例如下面的例子:
\begin{lstlisting}
#include <stdio.h>
#include <stdlib.h>

int main(){
    int a = 0;
    scanf("%d", &a);
    if(a % 2 == 1){
        printf("odd!\n");
    }else if(a % 2 == 0){
        printf("even!\n");
    }
    
    system("pause");

    return 0;
}
\end{lstlisting}

        我们又注意到一个数模2的结果不是1就是0, 故第二个\texttt{if}可以省略, 直接使用\texttt{else}. 例如下面的例子:
\begin{lstlisting}
#include <stdio.h>
#include <stdlib.h>

int main(){
    int a = 0;
    scanf("%d", &a);
    if(a % 2 == 1){
        printf("odd!\n");
    }else{
        printf("even!\n");
    }
    
    system("pause");

    return 0;
}
\end{lstlisting}

        而\texttt{if}语句其实只关心括号中表达式的值, 当值为\texttt{1}(真)时就执行, 反之为\texttt{0}(假)时就不执行, 所以虽然有些反直觉, 但上面的代码等价于下面的例子:
\begin{lstlisting}
#include <stdio.h>
#include <stdlib.h>

int main(){
    int a = 0;
    scanf("%d", &a);
    if(a % 2){
        printf("odd!\n");
    }else{
        printf("even!\n");
    }
    
    system("pause");

    return 0;
}
\end{lstlisting}

        虽然到了这一步我们已经既\textbf{不推荐}, 也没有必要了.\footnote{因为现代编译器非常智能, 会自动帮我们把生成的可执行文件中的命令优化成这个样子, 在源代码文件中最要紧的还是保证可读性.} 但这仍然是很好的思维训练, 能够帮助我们理解代码执行的原理. 结合第\ref{运算}节, 想想这是为什么.

        \begin{mdframed}[linecolor=darkgray]
        此外, 我们还可以用第\ref{位运算}节中提到的位运算来加速这个判断. 因为我们注意到, 根据式\ref{进制转换公式}, 一个数是奇数, 等价于它的二进制表达式第0位是1; 反之, 一个数是偶数, 等价于它的二进制表达式第0位是0. 想想这是为什么. 利用这个性质, 我们把\texttt{a}按位与1, 由于1的第0位是1, 1与任何数求与运算等于这个数本身, 故结果的第0位为\texttt{a}的第0位, 而1的其它位都是0, 故结果的其它位一定都是0. 因此按位与的结果就是\texttt{a}的第0位. 故我们可以写出下面的代码:
\begin{lstlisting}
#include <stdio.h>
#include <stdlib.h>

int main(){
    int a = 0;
    scanf("%d", &a);
    if(a & 1){
        printf("odd!\n");
    }else{
        printf("even!\n");
    }
    
    system("pause");

    return 0;
}
\end{lstlisting}
        \end{mdframed}

        读者可以自行验证上面这些代码的正确性.

    \section{循环结构} \label{循环结构}
        分支结构指的是程序运行到此处时, 会在我们定义的逻辑条件为真时反复运行, 直到条件变为假. 这在C语言中由两种语句实现.

        \subsection*{\texttt{while}语句}
            \texttt{\texttt{while}}语句的语法形如
            \begin{center}
            \begin{longtable}{l}
                \texttt{while(条件)\{} \\
                \qquad \texttt{语句块} \\
                \texttt{\}} 
            \end{longtable}
            \end{center}
            
            运行到\texttt{while}一行时, 将判断 \texttt{条件} 的真假, 若为假则跳过 \texttt{语句块} , 继续运行; 若为真则执行\texttt{语句块}, 然后跳回\texttt{while}一行再次判断, 重复此过程直到\texttt{条件}为假.

            例如下面的例子:
\begin{lstlisting}
#include <stdio.h>
#include <stdlib.h>

int main(){
    int a = 0;

    while(a < 5){
        printf("%d ", a);
        a++;
    }

    printf("\n");
    system("pause");

    return 0;
}
\end{lstlisting}

            第1次执行到\texttt{while}行, \texttt{a < 5}成立, 故打印\texttt{a}, 然后\texttt{a}自增1变为1, 语句块结束, 再次回到\texttt{while}行, \texttt{a < 5}成立, 再次执行代码块 \dots 如此这般直到执行代码块五次后, 此时\texttt{a}值为5, 第6次执行\texttt{while}行, \texttt{a < 5}不成立, 结束循环, 打印回车.

            故输出为
\lstset{
    numbers=none,
    keywordstyle=\color[RGB]{0,0,0},
    commentstyle=\color[RGB]{0,0,0},
    stringstyle=\color[RGB]{0,0,0}
}
\begin{lstlisting}
0 1 2 3 4 
\end{lstlisting}
\lstset{
    numbers=left,
    keywordstyle=\color[RGB]{3,95,205},
    commentstyle=\color[RGB]{34,139,34},
    stringstyle=\color[RGB]{128,0,0}
}

            我们也可以使用语法
            \begin{center}
            \begin{longtable}{l}
                \texttt{do\{} \\
                \qquad \texttt{语句块} \\
                \texttt{\}while(条件);}
            \end{longtable}
            \end{center}
            来把条件判断放在语句块执行之后. 注意这种写法\texttt{while()}后需要有一个分号. 二者的区别在于\texttt{do while}语句即使\texttt{条件}不成立, 也会在第一次执行到\texttt{do}行时, 进入循环. 换言之\texttt{do while}语句中的\texttt{语句块}至少会被执行一次.

            此外, 我们也可以用\texttt{while(1)}来构造一个死循环. 因为条件恒为真, 这个循环将一直执行下去. 例如下面的例子:
\begin{lstlisting}
#include <stdio.h>
#include <stdlib.h>

int main(){
    while(1){
        printf("Ha");
    }

    return 0;
}
\end{lstlisting}

            这个程序不会自己结束, 只能强制关闭. 读者也可以打开任务管理器查看性能, 读者应该会看到CPU中的某一个核心为了这段代码满载运行了.

            \subsection*{\texttt{for}语句}
            \texttt{for}语句也可以看作\texttt{while}语句的语法糖. 它的语法形如
            \begin{center}
            \begin{longtable}{l}
                \texttt{for(语句\hspace*{-0.25pt}1; 条件; 语句\hspace*{-0.25pt}2)\{} \\
                \qquad \texttt{语句块} \\
                \texttt{\}}       
            \end{longtable}
            \end{center}

            它等价于
            \begin{center}
            \begin{longtable}{l}
                \texttt{语句\hspace*{-0.25pt}1} \\ 
                \texttt{while(条件)\{} \\
                \qquad \texttt{语句块} \\
                \qquad \texttt{语句\hspace*{-0.25pt}2} \\
                \texttt{\}} \\
            \end{longtable}
            \end{center}

            也就相当于 \texttt{语句1} 是初始化, \texttt{语句2}是每次循环后执行的固定命令.

            但小区别在于, \texttt{for}中的 \texttt{语句1} 中声明的变量会在 \texttt{for}语句结束后被回收, 此时重新声明不会引发错误. 但\texttt{while}语句中的 \texttt{语句1} 则不会被回收, 不能在后文中重新声明, 对于一些常用的暂时变量, 这可能引发麻烦. 详见第\ref{变量的作用域}节. 因此我们\textbf{建议}:
            \begin{itemize}
                \item 诸如\texttt{i}, \texttt{j}, \texttt{k}这样的暂用的循环变量, 只在使用时在\texttt{for}语句中声明.
            \end{itemize}

            利用\texttt{for}语句, 我们可以使代码更紧凑, 表意更清晰. 例如前文中的代码可以写作:
\begin{lstlisting}
#include <stdio.h>
#include <stdlib.h>

int main(){
    for (int a = 0; a < 5; a++){
        printf("%d ", a);
    }

    printf("\n");
    system("pause");

    return 0;
}
\end{lstlisting}

            此外, 因为\texttt{a}被声明在了\texttt{for}语句中, 语句外声明\texttt{a}的代码是合法的:
\begin{lstlisting}
#include <stdio.h>
#include <stdlib.h>

int main(){
    for (int a = 0; a < 5; a++){
        printf("%d ", a);
    }

    int a = 0;
    a -= 5;

    printf("\n");
    system("pause");

    return 0;
}
\end{lstlisting}

            而使用\texttt{while}则会报错.

            我们再展示一个例子, 来说明循环结构的应用: 给定正数\texttt{n}, 求式子
                \[ \sum^ n _ {i = 1} \, \, i ^ 2 \]
            的值.

            它的实现有些微妙, 相比之下言语有些苍白. 我们直接贴出代码, 读者自行从代码的语义出发, 想想这是为什么.
\begin{lstlisting}
#include <stdio.h>
#include <stdlib.h>

int main(){
    unsigned long res = 0;
    int n = 0;
    scanf("%d", &n);

    for(int i = 1; i <= n; i++){
        res += i * i;
    }

    printf("%lu\n", res);
    system("pause");

    return 0;
}
\end{lstlisting}

            \begin{sloppypar}
            注意, 由于结果可能很大, 同时我们能保证它是非负的, 所以我们应该把它的类型声明为\texttt{unsigned long}.
            \end{sloppypar}

            其实利用数学归纳法容易证明
                \[ \sum ^n _{i = 1} \,\, i ^ 2 = \dfrac{n(n + 1)(2n + 1)}{6} .\]

            也就是说其实上面的代码等价于:
\begin{lstlisting}
#include <stdio.h>
#include <stdlib.h>

int main(){
    int n = 0;
    scanf("%d", &n);

    unsigned long res = n * (n + 1) * (2 * n + 1) / 6;

    printf("%lu\n", res);
    system("pause");

    return 0;
}
\end{lstlisting}

            这样我们大大优化了这段代码, 运行效率大大提高了. 这也某种意义上说明了数学和计算机科学的紧密联系, 和写代码前分析问题的重要性. 当然, 我们的本意只是介绍求和式计算的范式.

            读者也可以想想:
            \begin{enumerate}
                \item 第8行代码能不能写作\texttt{unsigned long res = n / 6 * (n + 1) * (2 * n + 1);} ? 为什么? 怎样改正这种写法的错误之处? 改正后相较于第8行的写法有什么优势? (答案见\ref{平方和公式答案})
                \item 求积式该如何用代码求出? 例如 $\displaystyle \prod ^n _{i = 1}\,\, i$
            \end{enumerate}

            此外, 和\texttt{if}相似, 当\texttt{while}和\texttt{for}后的语句块中只有一个语句时, 大括号可以省略, 但我们\textbf{不推荐}这么做.

    \section{\texttt{break}和\texttt{continue}}
            C语言定义了两个关键字\texttt{break}和\texttt{continue}, 用于循环结构的操控(此外, 正如我们学过的\texttt{break}语句也能用在\texttt{switch}语句中). \texttt{break}的语法形如
            \begin{center}
            \begin{longtable}{l}
                \texttt{break;} 
            \end{longtable}
            \end{center}
            \texttt{continue}的语法形如
            \begin{center}
            \begin{longtable}{l}
                \texttt{continue;} 
            \end{longtable}
            \end{center}

            \texttt{break}语句能跳出整个循环, 继续执行后面的代码, 例如下面的例子:
\begin{lstlisting}
#include <stdio.h>
#include <stdlib.h>

int main(){
    for(int i = 0; i < 5; i++){
        if(i == 3){
            break;
        }
        printf("%d ", i);
    }

    printf("\n");
    system("pause");

    return 0;
}
\end{lstlisting}

            前几次循环中\texttt{i == 3}为假, 故只打印\texttt{i}, 直到循环执行到 i = 3 时, 程序进入\texttt{if}语句内, \texttt{break}执行, 跳出了\texttt{for}循环, 打印换行. 故输出为
\lstset{
    numbers=none,
    keywordstyle=\color[RGB]{0,0,0},
    commentstyle=\color[RGB]{0,0,0},
    stringstyle=\color[RGB]{0,0,0}
}
\begin{lstlisting}
0 1 2 
\end{lstlisting}
\lstset{
    numbers=left,
    keywordstyle=\color[RGB]{3,95,205},
    commentstyle=\color[RGB]{34,139,34},
    stringstyle=\color[RGB]{128,0,0}
}

            而\texttt{continue}则是结束此次循环, 开始下次循环(但\texttt{for}语句括号中的最后一个语句仍然会执行). 例如下面的例子:
\begin{lstlisting}
#include <stdio.h>
#include <stdlib.h>

int main(){
    for(int i = 0; i < 5; i++){
        if(i == 3){
            continue;
        }
        printf("%d ", i);
    }

    printf("\n");
    system("pause");

    return 0;
}
\end{lstlisting}

            i = 3时, 进入\texttt{if}语句, 执行\texttt{continue;} , 不再进行此次循环. 而别的循环则打印\texttt{i}, 故输出为:
\lstset{
    numbers=none,
    keywordstyle=\color[RGB]{0,0,0},
    commentstyle=\color[RGB]{0,0,0},
    stringstyle=\color[RGB]{0,0,0}
}
\begin{lstlisting}
0 1 2 4 
\end{lstlisting}
\lstset{
    numbers=left,
    keywordstyle=\color[RGB]{3,95,205},
    commentstyle=\color[RGB]{34,139,34},
    stringstyle=\color[RGB]{128,0,0}
}

    \section{多层循环, 循环与分支}
        循环可以嵌套, 分支也可以嵌套, 循环和分支也可以相互嵌套. 例如下面的例子:
\begin{lstlisting}
#include <stdio.h>
#include <stdlib.h>

int main(){
    printf("\t\tj is even\t\tj is odd\n");
    for(int i = 0; i < 5; i++){
        if(i % 2 == 0){
            printf("i is even\t");
        }else{
            printf("i is odd\t");
        }
        for(int j = 0; j < 2; j++){
            printf("(i, j) = (%d, %d)\t\t", i, j);
        }
        printf("\n");
    }

    printf("\n");
    system("pause");

    return 0;
}
\end{lstlisting}

        这段代码虽然繁琐, 但并不复杂. 读者可以结合输出理解, 并学习控制台中的排版技巧.

    \section{实例: 鸡兔同笼问题}
        我们研究一个例子, 来观察嵌套结构的应用.

        \begin{center}
            今有鸡兔混杂一笼, 数之, 有头n个, 有脚m只, 问鸡数x, 兔数y各几何?
        \end{center}

        相信大家对这个《孙子算经》中的问题都很熟了. 我们今天抛去故有的思路, 从零开始, 试着用程序解决它.

        我们知道问题对应求解方程组
        \begin{align}
            \left\{
                \begin{array}{l}
                    x + y = n \\
                    2x + 4y = m
                \end{array}
            \right. \tag{1} \\
            x, y, n, m  \in \mathbb{N}  \notag
        \end{align} \label{鸡兔同笼方程}

        首先我们先尝试最朴素的方法: 暴力穷举. 计算机的强大之处就在于超快的计算力, 所以穷举虽然看似朴素甚至愚蠢, 但常常是解决计算机问题时最容易写出代码的办法, 甚至有时在处理一些极端复杂的问题时, 穷举是唯一的方法. 
        
        我们考虑设置两个循环变量\texttt{i}和\texttt{j}, 分别表示鸡的数量和兔子的数量, 令它们遍历 $ [0, a] \times [0, a]$中的所有整点(即尝试所有(i, j)的可能组合), 当它们满足式\ref{鸡兔同笼方程}时, 输出\texttt{i}和\texttt{j}, 即为解. 例如下面的例子:
\begin{lstlisting}
#include <stdio.h>
#include <stdlib.h>

int main(){
    int n = 0, m = 0;
    scanf("%d %d", &n, &m);

    for(int i = 0; i <= n; i++){
        for(int j = 0; j <= m; j++){
            if(i + j == n && 2 * i + 4 * j == m){
                printf("%d %d", i, j);
            }
        }
    }

    printf("\n");
    system("pause");

    return 0;
}
\end{lstlisting}

        这样我们就完成了一个在有解的情况下可以给出正确答案的代码雏形了.
        
        我们再观察式\ref{鸡兔同笼方程}, 由于它是非退化线性方程组, 故至多有一组解. 也就是说, 当我们找到一组解时, 就可以停止穷举了. 我们可以通过设置一个新变量\texttt{flag}来实现这一点, 他表示我们是否找到解了, 它被初始化为0, 当我们找到一组解时, 它被赋值为1, 而循环条件中加上\texttt{flag}为0. 我们还可以通过这个变量来判断有没有解, 若循环结束它仍然为0, 我们就可以断言输入的问题无解. 例如下面的例子:
\begin{lstlisting}
#include <stdio.h>
#include <stdlib.h>

int main(){
    int n = 0, m = 0;
    scanf("%d %d", &n, &m);

    int flag =  0;

    for(int i = 0; i <= n && flag == 0; i++){
        for(int j = 0; j <= m && flag == 0; j++){
            if(i + j == n && 2 * i + 4 * j == m){
                printf("%d %d", i, j);
                flag = 1;
            }
        }
    }

    if(flag == 0){
        printf("No solution exists!");
    }

    printf("\n");
    system("pause");

    return 0;
}
\end{lstlisting}

        这个代码看似很好了, 既能在有解时得出解, 又能在无解时判断出无解, 然而读者可以尝试输入
\lstset{
    numbers=none,
    keywordstyle=\color[RGB]{0,0,0},
    commentstyle=\color[RGB]{0,0,0},
    stringstyle=\color[RGB]{0,0,0}
}
\begin{lstlisting}
100000 300000
\end{lstlisting}
\lstset{
    numbers=left,
    keywordstyle=\color[RGB]{3,95,205},
    commentstyle=\color[RGB]{34,139,34},
    stringstyle=\color[RGB]{128,0,0}
}

        我们可以看到, 程序运行了很久才输出了答案(5000, 5000). 前面的章节中所有的代码都很简单, 运行都在零点零几微妙内就完成了, 所以读者可能忽略了代码的执行其实是需要时间的. 一般情况下, CPU一秒钟可以执行1e8 $\sim$ 1e9行C语言代码, 而我们注意在输入100000 300000时, 外循环执行了1e5次, 每次外循环由执行了一次内循环, 而每次内循环时, 内循环内的代码又被执行了1e5次. 故内循环内的代码需要被执行$1e5 \times 1e5 = 1e10$次!(有解时会中断循环, 故实际上小于这个值, 但数量级是不变的) 所以理论上这个简单的程序需要10s到100s不等. 我们迫切需要优化代码.

        我们注意到, (i, j)是解时, 一定有$i + j = n$, 故我们其实可以省去内层循环, 把\texttt{j}用\texttt{n - i}表示, 那么我们的代码变为:
\begin{lstlisting}
#include <stdio.h>
#include <stdlib.h>

int main(){
    int n = 0, m = 0;
    scanf("%d %d", &n, &m);

    int flag =  0;

    for(int i = 0; i <= n && flag == 0; i++){
        int j = n - i;
            if(2 * i + 4 * j == m){
                printf("%d %d", i, j);
                flag = 1;
            }
    }

    if(flag == 0){
        printf("No solution exists!");
    }

    printf("\n");
    system("pause");

    return 0;
}
\end{lstlisting}

        此时我们若再次尝试刚刚的输入, 我们只有一层循环了, 那么判断语句这一次只会被执行1e5次, 而1e5相较于1e8, 不过沧海之一粟. 所以这一次我们的程序又``立即''得出结果了. 然而如果我们再输入更大的数, 程序还是需要运行很久(虽然可能在此之前已经溢出了).

        当然, 我们都学过数学, 我们可以直接解出式\ref{鸡兔同笼方程}的解的表达式为
        \begin{align}
            \left\{
                \begin{array}{l}
                    x = 2n - \dfrac{m}{2} \\
                    y = \dfrac{m}{2} - n
                \end{array}
            \right. \tag{2}
        \end{align}

        注意到, x, y为整数的条件是m为偶数, 同时还必须满足x, y非负. 故把条件和表达式翻译成程序之后, 我们写出:
\begin{lstlisting}
#include <stdio.h>
#include <stdlib.h>

int main(){
    int n = 0, m = 0;
    scanf("%d %d", &n, &m);

    int x = 2 * n - m / 2;
    int y = m / 2 - n;

    if(x < 0 || y < 0 || m % 2 == 1){
        printf("No solution exists!");
    }else{
        printf("%d %d", x, y);
    }


    printf("\n");
    system("pause");

    return 0;
}
\end{lstlisting}

        这一次, 无论我们输入的数据多大, 只要不溢出, 程序都能在固定的很短的时间内给出答案了. 这也再一次应证了数学在计算机科学中的重要性.

        总之, 多层循环是危险的, 它可能会使一个简单的任务占用大量的时间, 造成严重的性能下降. 在非必要的情况下, 我们\textbf{建议}尽可能把循环嵌套层数降下来, 零层循环(顺序结构)就能解决问题是最好的. 当然, 这需要技巧, 而且也并非每个任务都能优化掉必要的循环的.

    \section{\texttt{*}条件运算符\texttt{*}} \label{条件运算符}
        条件运算符\texttt{?:}是一个三目运算符(也是C语言中唯一一个三目运算符, 故有时也直接用``三目运算符''来指代), 用法形如
            \[ \mbox{\texttt{表达式\vspace*{-1pt}1? 表达式\vspace*{-1pt}2 : 表达式\vspace*{-1pt}3}} \]

        当 \texttt{表达式\vspace*{-1pt}1} 为真时, 它返回 \texttt{表达式\vspace*{-1pt}2} , 否则返回 \texttt{表达式\vspace*{-1pt}3} .

        例如
\begin{lstlisting}
int main(){
    int a = 10, b = 5;

    int max = (a > b)? a : b;

    return 0;
}
\end{lstlisting}
        等价于
\begin{lstlisting}
int main(){
    int a = 10, b = 5;

    int max;
    if(a > b){
        max = a;
    }else{
        max = b;
    }

    return 0;
}
\end{lstlisting}

        条件运算符常用于压缩代码.

        条件运算符是右结合的, 也可以嵌套使用, 但我们\textbf{不推荐}这么做.

    \section{\texttt{*goto}语句\texttt{*}}
        \texttt{goto}语句是汇编时代的遗珍, 但由于容易导致恶行BUG, 并且很难维护, 现在几乎不再被使用, 在高级语言中也几乎都被禁用. C语言因为发明的早, 仍然保留着这个语句. 我们虽然不能一棒子打死说\texttt{goto}语句就是不好的, 例如Linux内核中就大量使用\texttt{goto}语句, 提高性能和可读性, 但总的而言它很难被驾驭, 我们\textbf{不推荐}使用. 它的语法形如
        \begin{center}
        \begin{longtable}{l}
            \texttt{label:} \\
            \texttt{goto label;}
        \end{longtable}
        \end{center}

        其中\texttt{label}是程序员自定的, 可以放在程序中的任意位置(独立成行, 或冒号后接其它语句). 执行\texttt{goto label;}后, 程序无条件跳转到\texttt{label}处继续运行. 本质上循环和分支结构都是\texttt{goto}语句实现的, 但有了高级语言的封装, 我们不再需要去操心底层的跳转指令了.