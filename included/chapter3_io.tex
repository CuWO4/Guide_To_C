\chapter{输入和输出} \label{输入和输出}
    我们的程序可以和程序外的用户(或其它程序模拟的用户)交互, 交互的方式就是交换信息, 包括输入信息和输出信息(也称打印信息). 本章就介绍各类与命令行窗口相关的输入输出.

    \section{\texttt{printf()}和\texttt{scanf()}}
        \begin{sloppypar}
        C语言标准库\texttt{stdio.h}中定义了两个强大的输入输出函数\footnote{函数的概念不同于数学上的函数, 指的是根据输入的参数, 执行一定功能的东西. 具体将在第\ref{函数}章中讨论.}\texttt{printf()}\footnote{PRINT Format, 格式化输出的缩写}和\texttt{scanf()}\footnote{SCAN Format, 格式化输入的缩写}. 它们可以实现在控制台中格式化输入输出文本, 变量等内容. 我们先介绍\texttt{printf()}的输出功能.
        \end{sloppypar}

        \subsection*{\texttt{printf()}的格式}
            printf()的格式形如:
                \[ \mbox{\texttt{printf("\hspace*{-0.25pt}文本", 参数1, 参数2, ...)}} \]
            \texttt{文本}可以是空文本. \texttt{参数}可以有若干个, 也可以没有.

            当没有参数时, \texttt{printf()}将直接输出文本内容. 例如下面的例子:
\begin{lstlisting}
#include <stdio.h>
#include <stdlib.h>

int main(){
    printf("Hello World!\n");
    system("pause");

    return 0;
}
\end{lstlisting}
            会输出
\lstset{
    numbers=none,
    keywordstyle=\color[RGB]{0,0,0},
    commentstyle=\color[RGB]{0,0,0},
    stringstyle=\color[RGB]{0,0,0}
}
\begin{lstlisting}
Hello World!
\end{lstlisting}
\lstset{
    numbers=left,
    keywordstyle=\color[RGB]{3,95,205},
    commentstyle=\color[RGB]{34,139,34},
    stringstyle=\color[RGB]{128,0,0}
}

            参数总是和文本中的格式符一一对应. 格式符以\texttt{\%}开头, 用一个字母表示参数的类型, 例如\texttt{\%d}表示十进制\texttt{int}型, \texttt{\%f}表示\texttt{float}型. 在文本中检测到格式符后, \texttt{printf()}将把对应顺序的参数带入到格式符中. 例如下面的例子:
\begin{lstlisting}
#include <stdio.h>
#include <stdlib.h>

int main(){
    int a = 5;
    float b = 10;
    printf("%d\n", a);        // 输出a并换行.
    printf("%d, %f\n", a, b); // 输出a, 用逗号间隔后再输出b, 然后换行.
    system("pause");

    return 0;
}
\end{lstlisting}

            参数的类型和格式符的输出类型应该一一对应, 否则会导致意料之外的错误. 参数的数量和格式符的数量应该一一对应, 当参数数量多于格式符数量时, 程序会忽略超过格式符数量的参数; 当格式符数量多于参数数量时, 多余参数数量的格式符的输出会发生意料之外的错误. 这些情况编译器一般都会警告.

            另外, 当我们要直接输出\texttt{\%}符号时, 为防止编译器把它和后面的符号误认为格式符, 我们需要把它转义. 例如下面的例子:
\begin{lstlisting}
#include <stdio.h>
#include <stdlib.h>

int main(){
    printf("\%d"); // 直接输出文本%d.
    system("pause");

    return 0;
}
\end{lstlisting}


        \subsection*{\texttt{printf()}输出整型}
            十进制\texttt{int}的格式符为\texttt{\%d}, 当有\texttt{short}修饰时, 加上\texttt{h}前缀; 当有\texttt{long}修饰时, 加上\texttt{\%l}修饰. 例如下面的例子:
\begin{lstlisting}
#include <stdio.h>
#include <stdlib.h>

int main(){
    int a = 10;
    short b = 10;
    long c = 10;
    printf("%d, %hd, %ld\n", a, b, c);
    system("pause");

    return 0;
}
\end{lstlisting}
            将输出
\lstset{
    numbers=none,
    keywordstyle=\color[RGB]{0,0,0},
    commentstyle=\color[RGB]{0,0,0},
    stringstyle=\color[RGB]{0,0,0}
}
\begin{lstlisting}
10, 10, 10
\end{lstlisting}
\lstset{
    numbers=left,
    keywordstyle=\color[RGB]{3,95,205},
    commentstyle=\color[RGB]{34,139,34},
    stringstyle=\color[RGB]{128,0,0}
}

            如果错误地指定变量的长度, 可能不会发生错误, 也可能发生意料之外的错误.

            默认情况下, 正数省去正号, 负数输出负号. 我们也可以加上\texttt{+}前缀, 来强制输出数的符号. 例如下面的例子:
\begin{lstlisting}
#include <stdio.h>
#include <stdlib.h>

int main(){
    int a = -10;
    int b = 10;
    printf("%d, %d, %+d, %+d\n", a, b, a, b);
    system("pause");

    return 0;
}
\end{lstlisting}
            将输出
\lstset{
    numbers=none,
    keywordstyle=\color[RGB]{0,0,0},
    commentstyle=\color[RGB]{0,0,0},
    stringstyle=\color[RGB]{0,0,0}
}
\begin{lstlisting}
-10, 10, -10, +10
\end{lstlisting}
\lstset{
    numbers=left,
    keywordstyle=\color[RGB]{3,95,205},
    commentstyle=\color[RGB]{34,139,34},
    stringstyle=\color[RGB]{128,0,0}
}

            另外我们可以加上空格前缀, 使得输出负数时正常输出, 输出正数时输出一个空格, 来帮助对齐格式. 例如下面的例子:
\begin{lstlisting}
#include <stdio.h>
#include <stdlib.h>

int main(){
    int a = 10;
    int b = -10;
    printf("%d\n%d\n", a, b);
    printf("\n");
    printf("% d\n% d\n", a, b);
    system("pause");

    return 0;
}
\end{lstlisting}
            的输出为
\lstset{
    numbers=none,
    keywordstyle=\color[RGB]{0,0,0},
    commentstyle=\color[RGB]{0,0,0},
    stringstyle=\color[RGB]{0,0,0}
}
\begin{lstlisting}
10
-10

 10
-10
\end{lstlisting}
\lstset{
    numbers=left,
    keywordstyle=\color[RGB]{3,95,205},
    commentstyle=\color[RGB]{34,139,34},
    stringstyle=\color[RGB]{128,0,0}
}
            
            其中前两行数字没有对齐, 倒数两行10前补了一个空格, 数字时对齐的.

            如果我们想输出无符号数, 那么就使用\texttt{\%u}格式符. 对于\texttt{short}和\texttt{long}我们同样加上\texttt{h}和\texttt{l}前缀. 例如下面的例子:
\begin{lstlisting}
#include <stdio.h>
#include <stdlib.h>

int main(){
    unsigned int a = 10;
    unsigned short b = 10;
    unsigned long c = 10;
    printf("%u, %hu, %lu\n", a, b, c);
    system("pause");

    return 0;
}
\end{lstlisting}
            将输出
\lstset{
    numbers=none,
    keywordstyle=\color[RGB]{0,0,0},
    commentstyle=\color[RGB]{0,0,0},
    stringstyle=\color[RGB]{0,0,0}
}
\begin{lstlisting}
10, 10, 10
\end{lstlisting}
\lstset{
    numbers=left,
    keywordstyle=\color[RGB]{3,95,205},
    commentstyle=\color[RGB]{34,139,34},
    stringstyle=\color[RGB]{128,0,0}
}

            如果用\texttt{\%u}输出负数会发生意料之外的错误.

            如果我们想以八进制输出, 就把\texttt{\%d}换为\texttt{\%o}; 同理如果想以十六进制输出, 就换为\texttt{\%x}或\texttt{\%X}(区别在与字母的大小写). 对于\texttt{short}和\texttt{long}我们同样加上\texttt{h}和\texttt{l}前缀. 例如下面的例子:
\begin{lstlisting}
#include <stdio.h>
#include <stdlib.h>

int main(){
    int a = 10;
    short b = 10;
    long c = 10;
    printf("%d, %ho, %lx, %lX\n", a, b, c, c);
    system("pause");

    return 0;
}
\end{lstlisting}
            将输出
\lstset{
    numbers=none,
    keywordstyle=\color[RGB]{0,0,0},
    commentstyle=\color[RGB]{0,0,0},
    stringstyle=\color[RGB]{0,0,0}
}
\begin{lstlisting}
10, 12, a, A
\end{lstlisting}
\lstset{
    numbers=left,
    keywordstyle=\color[RGB]{3,95,205},
    commentstyle=\color[RGB]{34,139,34},
    stringstyle=\color[RGB]{128,0,0}
}

            另外需要注意, \texttt{\%o}, \texttt{\%x}和\texttt{\%X}只能输出无符号数.

            如果想加上前缀, 那么我们再加上\texttt{\#}前缀. 例如下面的例子:
\begin{lstlisting}
#include <stdio.h>
#include <stdlib.h>

int main(){
    int a = 10;
    short b = 10;
    long c = 10;
    printf("%d, %#ho, %#lx, %#lX\n", a, b, c, c);
    system("pause");

    return 0;
}
\end{lstlisting}
            将输出
\lstset{
    numbers=none,
    keywordstyle=\color[RGB]{0,0,0},
    commentstyle=\color[RGB]{0,0,0},
    stringstyle=\color[RGB]{0,0,0}
}
\begin{lstlisting}
10, 012, 0xa, 0XA
\end{lstlisting}
\lstset{
    numbers=left,
    keywordstyle=\color[RGB]{3,95,205},
    commentstyle=\color[RGB]{34,139,34},
    stringstyle=\color[RGB]{128,0,0}
}

            另外\texttt{\%\#d}是UB, 部分编译器会忽略\texttt{\#}当作\texttt{\%d}, 部分编译器会输出意料之外的错误, 部分编译器会开启井字棋.

        \subsection*{\texttt{printf()}输出浮点数}
            对于一个\texttt{float}型变量, 我们可以使用 \texttt{\%f} , \texttt{\%e} , \texttt{\%E} , \texttt{\%g} 和 \texttt{\%G} 五种控制符输出. 其中 \texttt{\%f} 是用十进制输出, \texttt{\%e} 使用 \texttt{aen} 形式输出, \texttt{\%g} 自动选择十进制或 \texttt{aen} 中短的那种格式输出, 并省去多余的0. \texttt{\%E} 和 \texttt{\%G} 与 \texttt{\%e} 和 \texttt{\%g} 用法相同, 只是它们使用 \texttt{aEn} 形式, 即 \texttt{E} 是大写的. 例如下面的例子:
\begin{lstlisting}
#include <stdio.h>
#include <stdlib.h>

int main(){
    float a = 0.000025;
    printf("%f\n", a);
    printf("%e\n%E\n", a, a);
    printf("%g\n%G\n", a, a);
    system("pause");

    return 0;
}
\end{lstlisting}
            的输出为
\lstset{
    numbers=none,
    keywordstyle=\color[RGB]{0,0,0},
    commentstyle=\color[RGB]{0,0,0},
    stringstyle=\color[RGB]{0,0,0}
}
\begin{lstlisting}
0.000025
2.500000e-005
2.500000E-005
2.5e-005
2.5E-005
\end{lstlisting}
\lstset{
    numbers=left,
    keywordstyle=\color[RGB]{3,95,205},
    commentstyle=\color[RGB]{34,139,34},
    stringstyle=\color[RGB]{128,0,0}
}

            对于\texttt{double}类型, 我们需要在前面加上\texttt{l}修饰. 例如下面的例子:
\begin{lstlisting}
#include <stdio.h>
#include <stdlib.h>

int main(){
    float a = 0.000025;
    printf("%lf\n", a);
    printf("%le\n%lE\n", a, a);
    printf("%lg\n%lG\n", a, a);
    system("pause");

    return 0;
}
\end{lstlisting}
            的输出为
\lstset{
    numbers=none,
    keywordstyle=\color[RGB]{0,0,0},
    commentstyle=\color[RGB]{0,0,0},
    stringstyle=\color[RGB]{0,0,0}
}
\begin{lstlisting}
0.000025
2.500000e-005
2.500000E-005
2.5e-005
2.5E-005
\end{lstlisting}
\lstset{
    numbers=left,
    keywordstyle=\color[RGB]{3,95,205},
    commentstyle=\color[RGB]{34,139,34},
    stringstyle=\color[RGB]{128,0,0}
}

            此外, 前文提到的\texttt{+}前缀和空格前缀对浮点数也适用. 例如下面的例子:
\begin{lstlisting}
#include <stdio.h>
#include <stdlib.h>

int main(){
    float a = 0.000025;
    printf("%+lf\n", a);
    printf("% lf\n", a);
    system("pause");

    return 0;
}
\end{lstlisting}
            的输出为
\lstset{
    numbers=none,
    keywordstyle=\color[RGB]{0,0,0},
    commentstyle=\color[RGB]{0,0,0},
    stringstyle=\color[RGB]{0,0,0}
}
\begin{lstlisting}
+0.000025
 0.000025
\end{lstlisting}
\lstset{
    numbers=left,
    keywordstyle=\color[RGB]{3,95,205},
    commentstyle=\color[RGB]{34,139,34},
    stringstyle=\color[RGB]{128,0,0}
}

        \subsection*{\texttt{printf()}输出字符和字符串}
            字符的控制符为\texttt{\%c}, 例如下面的例子:
\begin{lstlisting}
#include <stdio.h>
#include <stdlib.h>

int main(){
    char ch = 'a';
    printf("%cb%c", ch, 'c');

    return 0;
}
\end{lstlisting}
            的输出为
\lstset{
    numbers=none,
    keywordstyle=\color[RGB]{0,0,0},
    commentstyle=\color[RGB]{0,0,0},
    stringstyle=\color[RGB]{0,0,0}
}
\begin{lstlisting}
abc
\end{lstlisting}
\lstset{
    numbers=left,
    keywordstyle=\color[RGB]{3,95,205},
    commentstyle=\color[RGB]{34,139,34},
    stringstyle=\color[RGB]{128,0,0}
}

            此外, 由于字符本质是整型, 所以字符也可以用整型控制符输出, 反之亦然. 例如下面的例子:
\begin{lstlisting}
#include <stdio.h>
#include <stdlib.h>

int main(){
    char ch = 'a';
    int val = 97;
    printf("%d\n", ch);
    printf("%c\n", val);
    system("pause");

    return 0;
}
\end{lstlisting}
            的输出为
\lstset{
    numbers=none,
    keywordstyle=\color[RGB]{0,0,0},
    commentstyle=\color[RGB]{0,0,0},
    stringstyle=\color[RGB]{0,0,0}
}
\begin{lstlisting}
97
a
\end{lstlisting}
\lstset{
    numbers=left,
    keywordstyle=\color[RGB]{3,95,205},
    commentstyle=\color[RGB]{34,139,34},
    stringstyle=\color[RGB]{128,0,0}
}

        字符串的格式符为\texttt{\%s}. 例如下面的例子:
\begin{lstlisting}
#include <stdio.h>
#include <stdlib.h>

int main(){
    char *str = "Cool China";
    printf("%s!\n", str);
    system("pause");

    return 0;
}
\end{lstlisting}
        的输出为
\lstset{
    numbers=none,
    keywordstyle=\color[RGB]{0,0,0},
    commentstyle=\color[RGB]{0,0,0},
    stringstyle=\color[RGB]{0,0,0}
}
\begin{lstlisting}
Cool China!
\end{lstlisting}
\lstset{
    numbers=left,
    keywordstyle=\color[RGB]{3,95,205},
    commentstyle=\color[RGB]{34,139,34},
    stringstyle=\color[RGB]{128,0,0}
}

        \subsection*{格式化输出}
            我们可以在格式符后加一个数字, 表示这个变量输出的最小宽度, 不足的将在前面补空格. 例如下面的例子:
\begin{lstlisting}
#include <stdio.h>
#include <stdlib.h>

int main(){
    int a = 12, b = 5;
    printf("%5d%5d|\n", a, b);
    system("pause");

    return 0;
}
\end{lstlisting}
            的输出为
\lstset{
    numbers=none,
    keywordstyle=\color[RGB]{0,0,0},
    commentstyle=\color[RGB]{0,0,0},
    stringstyle=\color[RGB]{0,0,0}
}
\begin{lstlisting}
   12    5|
\end{lstlisting}
\lstset{
    numbers=left,
    keywordstyle=\color[RGB]{3,95,205},
    commentstyle=\color[RGB]{34,139,34},
    stringstyle=\color[RGB]{128,0,0}
}

            我们也可以再用\texttt{-}修饰, 表示左对齐. 例如下面的例子:
\begin{lstlisting}
#include <stdio.h>
#include <stdlib.h>

int main(){
    int a = 12, b = 5;
    printf("%-5d%-5d|\n", a, b);
    system("pause");

    return 0;
}
\end{lstlisting}
            的输出为
\lstset{
    numbers=none,
    keywordstyle=\color[RGB]{0,0,0},
    commentstyle=\color[RGB]{0,0,0},
    stringstyle=\color[RGB]{0,0,0}
}
\begin{lstlisting}
12   5    |
\end{lstlisting}
\lstset{
    numbers=left,
    keywordstyle=\color[RGB]{3,95,205},
    commentstyle=\color[RGB]{34,139,34},
    stringstyle=\color[RGB]{128,0,0}
}

            我们也可以用 \texttt{.} 后加一个数字修饰, 表示精确到小数点后几位, 不足的补零, 也可以和最小宽度连用. 例如下面的例子:
\begin{lstlisting} [caption=\ ,label=c_io_floatoutput]
#include <stdio.h>
#include <stdlib.h>

int main(){
    double a = 2.5;
    printf("%.2lf|\n", a);
    printf("%6.2lf|\n", a);
    printf("%-6.2lf|\n", a);
    system("pause");

    return 0;
}
\end{lstlisting}
            的输出为
\lstset{
    numbers=none,
    keywordstyle=\color[RGB]{0,0,0},
    commentstyle=\color[RGB]{0,0,0},
    stringstyle=\color[RGB]{0,0,0}
}
\begin{lstlisting}
2.50|
  2.50|
2.50  |
\end{lstlisting}
\lstset{
    numbers=left,
    keywordstyle=\color[RGB]{3,95,205},
    commentstyle=\color[RGB]{34,139,34},
    stringstyle=\color[RGB]{128,0,0}
}

            此外, 多行字符串之间不加符号, 编译器会自动帮我们把它们拼接在一起. 所以我们可以使用下面的技巧来增强\texttt{printf()}的可读性:
\begin{lstlisting}
#include <stdio.h>
#include <stdlib.h>

int main(){
    double a = 2.5;
    printf(
        "%.2lf|\n"
        "%6.2lf|\n"
        "%-6.2lf|\n",
        a,
        a,
        a
    );
    system("pause");

    return 0;
}
\end{lstlisting}

            这段代码和代码\ref{c_io_floatoutput}是等价的.

        \subsection*{\texttt{scanf()}}
            \texttt{scanf()}的基本格式与\texttt{printf()}类似, 形如
                \[ \mbox{\texttt{scanf("\hspace*{-0.25pt}文本", \&\hspace*{-0.25pt}参数\hspace*{-0.25pt}1, \&\hspace*{-0.25pt}参数\hspace*{-0.25pt}2,...)}} \]

            其中, \texttt{文本}中也可以插入控制符, 来吧对应位置的输入赋值给对应位置的\texttt{参数}, 控制符格式和\texttt{printf()}相同, 只是不能使用前缀修饰. 注意, \texttt{参数}前要加上\texttt{\&}, 原因将在第\ref{指针}章中解释.

            例如下面的例子:
\begin{lstlisting}
#include <stdio.h>
#include <stdlib.h>

int main(){
    int a = 0;
    double b = 0;
    scanf("%d, %lf", &a, &b);
    printf("%d, %lf\n", a, b);
    system("pause");

    return 0;
}
\end{lstlisting}

            用户输入的多个空白符会被当作一个空白符, 当用户输入的文本和\texttt{scanf()}中文本指定的格式不同时, 可能会发生意料之外的错误. 有时会发生奇怪的行为, 这与缓冲区有关, 我们不在这里讨论.

            特别地, 当输入字符串时, 我们反而不能用\texttt{\&}修饰, 并且字符串必须初始化过且输入的字符串不超过原来字符串的长度. 例如下面的例子:
\begin{lstlisting}
#include <stdio.h>
#include <stdlib.h>

int main(){
    char str[] = "Star War is brilliant.";
    scanf("%s", str);
    printf("%s\n", str);
    system("pause");

    return 0;
}
\end{lstlisting}

            原因将在第\ref{字符串}章中解释.

    \section{其它输入输出}
        \texttt{stdio.h}下还包含其它输入输出函数:

        \subsection*{\texttt{getchar()}}
            \texttt{getchar()}会读取一个字符并返回. 用法形如
                \[ \mbox{\texttt{c = getchar();}} \]
            
            等价于
                \[ \mbox{\texttt{scanf("\%c", \&c);}} \]  

        \subsection*{\texttt{putchar(char)}}
            \texttt{putchar(char)}会输出括号内的字符. 用法形如
                \[ \mbox{\texttt{putchar(c);}} \]

            等价于
                \[ \mbox{\texttt{printf("\%c", c);}} \]

        \subsection*{\texttt{gets(char*)}}
            \texttt{gets(char*)}需要把读取后的储存字符串写在其括号内, 它将读取一整行信息, 并存入指定的字符串中. 用法形如
                \[ \mbox{\texttt{gets(str);}} \]

            例如下面的例子:
\begin{lstlisting}
#include <stdio.h>
#include <stdlib.h>

int main(){
    char str[30];
    gets(str);
    printf("%s", str);
    system("pause");

    return 0;
}
\end{lstlisting}

            此外, 读者可以输入 Hello World , 比较上面的代码和下面的代码的效果:
\begin{lstlisting}
#include <stdio.h>
#include <stdlib.h>

int main(){
    char str[30];
    scanf("%s", str);
    printf("%s", str);
    system("pause");

    return 0;
}
\end{lstlisting}

            可以发现\texttt{gets(str);}和\texttt{scanf("\%s", str);}不完全等价. 前者会把空格当作字符串的一部分, 后者则会把空格当作字符串于其他部分的分割, 故而停止输入.

        \subsection*{\texttt{puts(char*)}}
            \texttt{puts(char*)}会输出括号内的字符串, 用法形如
                \[ \mbox{\texttt{puts(str);}} \]

            等价于
                \[ \mbox{\texttt{printf("\%s", str);}} \]

        \subsection*{ }
        此外, windows平台上还有一个只能使用于windows平台的库\texttt{conio.h}用于输入输出, 读者可以自行了解.