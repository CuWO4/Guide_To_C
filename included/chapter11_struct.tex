\chapter{结构体} \label{结构体}
    结构体是一种程序员自己定义的数据类型, 一个结构体变量中储存着若干个变量. 一个结构体的定义形如
    \begin{center}
    \begin{longtable}{l}
        \texttt{struct 结构体名\{} \\
        \qquad \texttt{内部变量声明} \\
        \texttt{\};}
    \end{longtable}
    \end{center}

    注意定义的大括号后面需要一个分号.

    当我们定义了一个结构体后, 就可以声明类型为这种结构体的变量, 语法形如
    \begin{center}
    \begin{longtable}{l}
        \texttt{struct 结构体名~变量名;}
    \end{longtable}
    \end{center}

    我们也可以在定义结构体时, 声明结构体变量, 语法形如
    \begin{center}
    \begin{longtable}{l}
        \texttt{struct 结构体名\{} \\
        \qquad \texttt{变量声明} \\
        \texttt{\}结构体变量名;}
    \end{longtable}
    \end{center}

    当我们声明了结构体变量后, 就可以用点运算符\texttt{.}来访问其中的变量, 语法形如
    \begin{center}
    \begin{longtable}{l}
        \texttt{结构体变量名.内部变量名}
    \end{longtable}
    \end{center}

    例如下面的例子:
\begin{lstlisting}
struct STU{
    int a;
    double b;
}x;

int main(){
    struct STU y;
    
    x.a = 1;
    x.b = 2.5;

    y.a = 0;
    y.b = 1.3;

    return 0;
}
\end{lstlisting}
    定义了一种结构体\texttt{STU}, 它的内部包含了整型的\texttt{a}和浮点型的\texttt{b}两个变量. 然后声明了两个类型为\text{STU}的变量\texttt{x}和\texttt{y}, 接着分别把其内部的\texttt{a}和\texttt{b}赋值为了1, 0和2.5, 1.3.

    结构体可以互相赋值, 例如下面的例子:
\begin{lstlisting}
struct STU{
    int a;
    double b;
};

int main(){
    struct STU x, y;

    x.a = 1;
    x.b = 2.5;

    y.a = 0;
    y.b = 1.3;
    
    struct STU temp = x;
    x = y;
    y = temp;

    return 0;
}
\end{lstlisting}
    就交换了\texttt{x}和\texttt{y}.

    我们也可以像基本的数据类型那样, 定义结构体数组和结构体指针, 它们的用法和基本数据类型的数组和指针相同, 在此不再赘述. 特别地若A为一个结构体指针, 那么
    \texttt{(*A).x}可以简写为\texttt{A->x}.

    结构体相当于把若干变量捆绑在一起, 当发生赋值等操作时, 这些变量会被一起赋值. 这在处理同一个对象的不同属性时很有用, 例如我们可以定义一个结构体\texttt{STUDENT}, 它包含学生姓名和学生成绩, 然后根据学生成绩对学生进行排序, 那么学生姓名也会随之一起被排序. 例如下面的例子:
\begin{lstlisting}
#include <stdio.h>
#include <stdlib.h>

#define N 3

struct STUDENT{
    int score;
    char *name;
}class_L9[N];

int init();
int print(struct STUDENT arr[], int len);
int swap(struct STUDENT *a, struct STUDENT *b);
int bubble_sort(struct STUDENT arr[], int len);

int main(){
    init();

    print(class_L9, N);
    printf("\n");

    bubble_sort(class_L9, N);

    print(class_L9, N);

    system("pause");

    return 0;
}

int init(){
    class_L9[0].score = 71;
    class_L9[0].name = "Andy";

    class_L9[1].score = 63;
    class_L9[1].name = "Bob";

    class_L9[2].score = 93;
    class_L9[2].name = "Carol";

    return 0;
}

int print(struct STUDENT arr[], int len){
    for(int i = 0; i < 3; i++){
        printf("%s: %d\n", arr[i].name, arr[i].score);
    }
    return 0;
}

int swap(struct STUDENT *a, struct STUDENT *b){
    struct STUDENT temp = *a;
    *a = *b;
    *b = temp;
    return 0;
}

int bubble_sort(struct STUDENT arr[], int len){
    if(len < 0){
        return -1;
    }

    while(1){
        int flag = 1;
        for(int i = 0; i < len - 1; i++){
            if(arr[i].score > arr[i + 1].score){
                swap(&arr[i], &arr[i + 1]);
                flag = 0;
            }
        }
        if(flag == 1){
            break;
        }
    }
    return 0;
}
\end{lstlisting}
    输出为
\lstset{
    numbers=none,
    keywordstyle=\color[RGB]{0,0,0},
    commentstyle=\color[RGB]{0,0,0},
    stringstyle=\color[RGB]{0,0,0}
}
\begin{lstlisting}
Andy: 71
Bob: 63
Carol: 93

Bob: 63
Andy: 71
Carol: 93
\end{lstlisting}
\lstset{
    numbers=left,
    keywordstyle=\color[RGB]{3,95,205},
    commentstyle=\color[RGB]{34,139,34},
    stringstyle=\color[RGB]{128,0,0}
}

    可见, 我们在\texttt{bubble\_sort()}中只对成绩进行了排序, 在结果中学生名字也随着成绩一同进行了排序.

    \begin{mdframed}[linecolor=gray]

    此外, 利用函数指针, 我们可以编写泛用性更广的排序. 例如下面的例子:
\begin{lstlisting}
int bubble_sort(struct STUDENT arr[], int len, int (*comp)(struct STUDENT a, struct STUDENT b)){
    if(len < 0){
        return -1;
    }

    while(1){
        int flag = 1;
        for(int i = 0; i < len - 1; i++){
            if(!comp(arr[i], arr[i + 1])){
                swap(&arr[i], &arr[i + 1]);
                flag = 0;
            }
        }
        if(flag == 1){
            break;
        }
    }
    return 0;
}
\end{lstlisting}

    其中, 形参\texttt{comp()}指定排序方法. 它的规则是, 对于一对相邻数据, 当有序时, 它返回真, 否则返回假.

    如果我们现在想实现按照首字母顺序排序, 那么我们只需要编写\texttt{comp2()}函数如下(默认首字母大写):
\begin{lstlisting}
    int comp2(struct STUDENT a, struct STUDENT b){
        if(a.name[0] < b.name[0]){
            return 1;
        }else{
            return 0;
        }
    }
\end{lstlisting}
    并在调用\texttt{bubble\_sort()}时使用:
\begin{lstlisting}
bubble_sort(class_L9, N, comp2);
\end{lstlisting}
    即可.   

    \end{mdframed}